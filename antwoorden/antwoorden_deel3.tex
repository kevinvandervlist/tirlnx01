% tirlnx01 - Materiaal om het keuzevak Linux te geven 
% op de Hogeschool Rotterdam.
% Copyright (C) 2010 - 2011  Paul Sohier, Kevin van der Vlist
%
% This program is free software: you can redistribute it and/or modify
% it under the terms of the GNU General Public License as published by
% the Free Software Foundation, either version 3 of the License, or
% (at your option) any later version.
%
% This program is distributed in the hope that it will be useful,
% but WITHOUT ANY WARRANTY; without even the implied warranty of
% MERCHANTABILITY or FITNESS FOR A PARTICULAR PURPOSE.  See the
% GNU General Public License for more details.
%
% You should have received a copy of the GNU General Public License
% along with this program.  If not, see <http://www.gnu.org/licenses/>.
%
% Kevin van der Vlist - kevin@kevinvandervlist.nl
% Paul Sohier - paul@paulsohier.nl

\section{Week 3}
Deze vragen gaan over de volgende hoofdstukken:
\begin{itemize}
\item[1.] User management
\item[2.] Proces management
\end{itemize}

\question[0] Kijk naar de volgende shell log:
\begin{lstlisting}
kevin@slackbak:~$ getent passwd root
root:x:0:0::/root:/bin/bash
kevin@slackbak:~$ getent passwd paul
paul:x:1001:100:,,,:/home/paul:/bin/bash
kevin@slackbak:~$ getent passwd kevin
kevin:x:1000:100:,,,:/home/kevin:/bin/bash
kevin@slackbak:~$ whoami
kevin
kevin@slackbak:~$ cp /usr/bin/top /tmp/top 
kevin@slackbak:~$ chmod 755 /tmp/top
kevin@slackbak:~$ chmod u+s /tmp/top
\end{lstlisting}%$

\begin{parts}
\part[15] Wat zijn nu de \emph{effective uid} en de \emph{real uid} van het volgende? Verklaar je antwoord.
\begin{lstlisting}
paul@slackbak:~$ /tmp/top 
\end{lstlisting}%$

\begin{solution}
\begin{lstlisting}
uid:  1001
euid: 1000
gid:  100
egid: 100
UID verhaal, sticky bit
\end{lstlisting}%$
\end{solution}

\part[15] Het proces van het programma \texttt{top} blijft actief. Wat gebeurt er nu met het \emph{effective uid} en het \emph{real uid} na de onderstaande stappen? En van nieuwe processen op \texttt{/tmp/top}?
\begin{lstlisting}
root@slackbak:~$ chown paul /tmp/top 
root@slackbak:~$ chmod u+s /tmp/top
\end{lstlisting}
\begin{solution}
\begin{lstlisting}
Actieve proces veranderd niet, zit al in memory. Nieuwe processen:
uid:  uid user
euid: 1001 (paul)
gid:  gid user
egid: egid user
\end{lstlisting}
\end{solution}
\end{parts}

\question[0] Bekijk de volgende situatie:
\begin{lstlisting}
kevin@slackbak:/tmp/map$ ls -lhR
.:
total 4.0K
drwxr--r-x 2 kevin users 4.0K 2011-01-16 13:28 map/

./map:
total 0
--w-r--r-- 1 kevin users 0 2011-01-16 13:28 file
\end{lstlisting}%$
\begin{parts}
\part[5] Mag de gebruiker kevin de map ``map'' in? En mag die het bestand ``file'' uitlezen?
\begin{solution}
map: ja, x-bit\\
file: ja, world
\end{solution}

\part[5] Mag een lid van de groep users de map ``map'' in? En mag die het bestand ``file'' uitlezen?
\begin{solution}
map: world x-bit, ja\\
file: ja, world
\end{solution}

\part[5] Mag een ander, dus ``other'', de map ``map'' in? En mag die het bestand ``file'' uitlezen?
\begin{solution}
map: ja, x-bit\\
file: ja, r-bit
\end{solution}
\end{parts}

\question[10] Bekijk welk \emph{pid} bij de actieve \texttt{bash} shell hoort. Kijk ook wat de parent is. Geef deze \emph{pid's} en de \emph{CMD's} van deze processen.
\begin{solution}
\begin{lstlisting}
kevin@slackbak:~$ ps -ef | grep bash
kevin     4413  4412  1 12:54 pts/0    00:00:00 -bash
kevin     4427  4413  0 12:54 pts/0    00:00:00 grep bash
kevin@slackbak:~$ ps -ef | grep 4412
kevin     4412  4409  0 12:54 ?        00:00:00 sshd: kevin@pts/0
kevin     4413  4412  0 12:54 pts/0    00:00:00 -bash
kevin     4448  4413  0 12:54 pts/0    00:00:00 grep 4451
# Note: ppid via console is 1
\end{lstlisting}
\end{solution}

\question[15] Stuur een \emph{SIGTERM} naar de \texttt{bash} shell. Wat gebeurt er? Stuur ook een \emph{SIGKILL} naar de \texttt{bash} shell, wat gebeurt er dan? Verklaar je antwoord.
\begin{solution}
\begin{lstlisting}
kevin@slackbak:~$ kill -15 4413
kevin@slackbak:~$ ps -ef | grep bash
kevin     4413  4412  0 12:54 pts/0    00:00:00 -bash
kevin     4448  4413  0 12:55 pts/0    00:00:00 grep bash
# SIGTERM wordt afgevangen - gebeurt niets
kevin@slackbak:~$ kill -9 4413
Connection to slack.icyx.nl closed
# Dus: bash sluit, waarna sshd zichzelf 'netjes' afsluit.
kevin@slackbak:~$ kill -15 4451Connection to slack.icyx.nl closed by remote host.
Connection to slack.icyx.nl closed.
# SIGKILL kan niet worden afgevangen, nu zal het proces aan de andere kant dood zijn. Sessie gaat dus kapot.
# Note: Op een console word login actief
\end{lstlisting}
\end{solution}

\question[0] Zombies:
\begin{parts}
\part[10] Wat is een zombie proces?
\begin{solution}
Een parent proces bekijkt de exit status van een child niet. Dit child proces blijft wachten met deze status, terwijl de parent al dood is. 
\end{solution}

\part[10] Waarom kan je deze niet killen?
\begin{solution}
ze krijgen \texttt{init} als parent
\end{solution}
\end{parts}

\question[10] Wie of wat heeft pid 0?
\begin{solution}
De kernel, ppid van init + kern threads = 0
\end{solution}

