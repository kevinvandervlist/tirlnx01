% tirlnx01 - Materiaal om het keuzevak Linux te geven 
% op de Hogeschool Rotterdam.
% Copyright (C) 2010 - 2011  Paul Sohier, Kevin van der Vlist
%
% This program is free software: you can redistribute it and/or modify
% it under the terms of the GNU General Public License as published by
% the Free Software Foundation, either version 3 of the License, or
% (at your option) any later version.
%
% This program is distributed in the hope that it will be useful,
% but WITHOUT ANY WARRANTY; without even the implied warranty of
% MERCHANTABILITY or FITNESS FOR A PARTICULAR PURPOSE.  See the
% GNU General Public License for more details.
%
% You should have received a copy of the GNU General Public License
% along with this program.  If not, see <http://www.gnu.org/licenses/>.
%
% Kevin van der Vlist - kevin@kevinvandervlist.nl
% Paul Sohier - paul@paulsohier.nl

\section{Week 4}
Deze vragen gaan over de volgende hoofdstukken:
\begin{itemize}
\item[1.] Linux networking
\item[2.] X Windows
\item[3.] Device files
\end{itemize}

\question[10] Geef het \emph{IP} en de interface naam van het netwerk device van de computer waar je nu op werkt. 
\begin{solution}
\begin{lstlisting}
kevin@slackbak:~$ /sbin/ifconfig 
eth0      Link encap:Ethernet  HWaddr 00:0c:29:fa:16:57  
          inet addr:145.24.222.162  Bcast:145.24.222.255  Mask:255.255.255.0
\end{lstlisting}%$
\end{solution}

\question[10] Geef het \emph{IP} adres en de \emph{RSA fingerprint} van de server van \emph{Slackware}, te vinden op ``slackware.org''. 
\begin{solution}
\begin{lstlisting}
kevin@slackbak:~$ ssh slackware.org
The authenticity of host 'slackware.org (168.150.251.105)' can't be established.
RSA key fingerprint is 87:92:7f:41:f2:4a:12:e7:93:97:70:85:cd:af:c8:b1.
Are you sure you want to continue connecting (yes/no)? ^C
\end{lstlisting}%$
\end{solution}

\question[10] Maak een bestand aan, en zet dit op de \emph{FTP} space van je studenten account. Zorg dat het in de ``public.www'' subdirectory staat. Tip: \emph{passive ftp} kan nodig zijn
\begin{solution}
\begin{lstlisting}
kevin@slackbak:~$ touch mijnupload
kevin@slackbak:~$ ftp -p ftp.hro.nl
Connected to orpheus.hro.nl.
[...]
220 Service Ready for new User
Name (ftp.hro.nl:kevin): 0814xxx.cmi
331 Password Needed for Login
Password:
230 User 0814xxx Logged in Successfully
Remote system type is NETWARE.
ftp> cd public.www
250 Directory successfully changed to "/USERS16/4/0814xxx/public.www"
ftp> put mijnupload bestand
local: mijnupload remote: bestand
200 PORT Command OK
150 Opening data connection for bestand (145.24.222.162,42544)
226 Transfer Complete
ftp> exit
221 Closing Session
\end{lstlisting}%$
\end{solution}

\question[10] Tijdens het programmeren van een stuk software is er een vereiste aan een hoge entropie om wachtwoorden te genereren. Onder geen beding mogen deze wachtwoorden voorspelbaar worden. Welk special device dient gebruikt te worden? Waarom?
\begin{solution}
random: blocking; hoge entropie\\
urandom: non-blocking; mogelijk voorspelbaar
\end{solution}

\question[10] Voeg de nameserver 2.4.8.16 toe. Geef de gedane stappen. 
\begin{solution}
\begin{lstlisting}
root@slackbak:/home/kevin# cat /etc/resolv.conf 
nameserver 2.4.8.16
\end{lstlisting}
\end{solution}

\question[10] Zorg dat wanneer er gezocht word naar de host \texttt{geheim.xyz}, dat dit resolved naar het \emph{IP} adres \texttt{1.2.3.4}. Geef de gedane stappen. 
\begin{solution}
\begin{lstlisting}
/etc/hosts
1.2.3.4          geheim.xyz
\end{lstlisting}
\end{solution}

\question[10] Maak een nieuw loopback device aan voor de \emph{IP} range \texttt{172.16.0.0}. Zorg dat de routing naar dit device verloo, en dat het adres \texttt{172.16.1.1} te pingen is. Verwijder de interface ook weer. Geef de commando's.
\begin{solution}
\begin{lstlisting}
ifconfig lo:0 172.16.0.0
route add 172.16.0.0 lo:0
ping 172.16.1.1
ifconfig lo:0 down
\end{lstlisting}
Note Paul: route lijkt niet nodig.
\end{solution}

\question[10] Verklaar de relatie tussen een X server en een X client. Leg uit wie de server, en wie de client is. Waarom is dit zo?
\begin{solution}

X server draait op de client, x client op de server (met networking) of client. Windows worden gepaint op de server, applicaties zijn zelf de client. Dit is gedaan vor de modulariteit en networking mogelijkheden
\end{solution}
\question[10] Veel window managers voor X zorgen dat applicaties als het ware in een frame worden geplaatst. Dit frame kan gebruikt worden voor de plaatsing van de window, maar bijvoorbeeld ook om verschillende buttons op te plaatsen. Hoe noemt men dit fenomeen?
\begin{solution}

Reparenting van windows
\end{solution}

\question[10] OPTIONEEL: Installeer een grafische omgeving, en configureer deze. Probeer dan eens van de desktop gebruik te maken. De voordelen van een krachtig, open systeem en een grafische omgeving voor standaard gebruik zullen dan duidelijk worden. Voor installatie van software, zie bijlage F en G van het dictaat.
\begin{solution}
Packages zijn het handigst
\end{solution}
