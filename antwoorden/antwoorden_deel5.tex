% tirlnx01 - Materiaal om het keuzevak Linux te geven 
% op de Hogeschool Rotterdam.
% Copyright (C) 2010 - 2011  Paul Sohier, Kevin van der Vlist
%
% This program is free software: you can redistribute it and/or modify
% it under the terms of the GNU General Public License as published by
% the Free Software Foundation, either version 3 of the License, or
% (at your option) any later version.
%
% This program is distributed in the hope that it will be useful,
% but WITHOUT ANY WARRANTY; without even the implied warranty of
% MERCHANTABILITY or FITNESS FOR A PARTICULAR PURPOSE.  See the
% GNU General Public License for more details.
%
% You should have received a copy of the GNU General Public License
% along with this program.  If not, see <http://www.gnu.org/licenses/>.
%
% Kevin van der Vlist - kevin@kevinvandervlist.nl
% Paul Sohier - paul@paulsohier.nl

\section{Week 5}
Deze vragen gaan over de volgende hoofdstukken:
\begin{itemize}
\item[1.] Basis commando's
\item[2.] Shell
\end{itemize}

\question[5] Maak je eigen opdracht met behulp van opdrachtalliassering. De opdracht moet naar je home-directory springen en daar de pico editorstarten.
\begin{solution}
\begin{lstlisting}
kevin@slackbak:~$ tail -n 1 ~/.bash_profile
alias homepico='cd ~; pico'
\end{lstlisting}%$
\end{solution}

\question[0] Haal zoveel mogelijk informatie uit de volgende \texttt{bash} prompts:
\begin{parts}
\part[5] 
\begin{lstlisting}
kevin@slackbak:~$ 
\end{lstlisting}%$

\begin{solution}
user kevin @ host slackbak in homedir zonder speciale rechten
\end{solution}

\part[5]
\begin{lstlisting}
root@1.2.3.4:/home/# 
\end{lstlisting}

\begin{solution}

user root @ host 1.2.3.4 in /home/ met admin rechten
\end{solution}
\end{parts}

\question[10] Zoek naar alle bestanden met de extensie .txt en pak ze in met gzip. Maak gebruik van opdracht substitutie. %todo, check locatie in dictaat
\begin{solution}
\begin{lstlisting}
find / -name '*.txt' -exec cat {} \; | gzip > files.gz
\end{lstlisting}
\end{solution}

\question[0] Deze vraag gaat over cronjobs.
\begin{parts}
\part[10] Geef aan welke cron jobs er iedere dag gerund worden op een standaard Slackware systeem.
\begin{solution}
\begin{lstlisting}
root@slackbak /var/log# ls /etc/cron.daily/
certwatch  logrotate  slocate
\end{lstlisting}
\end{solution}

\part[10] Een van de standaard crons is logrotate. Hierdoor worden log files verplaatst en ingepakt, waardoor het systeem weer met lege files kan beginnen. Er worden een aantal van deze ingepakte logs bewaard. Is het verstandig om log files lang te bewaren op een server? Waarom wel of niet?
\begin{solution}
Nee, schijfruimte gebruik is hoog op een server met grote logs. 

Ja, schijfruimte weegt niet op tegen de hoeveelheid informatie die je hebt. 
\end{solution}
\end{parts}

\question[0] Vul de onderstaande kolom in voor de volgende situaties:
\begin{parts}
\part[10] Een \texttt{chmod 651}
\part[10] Een \texttt{(chmod 400; chmod +x)}
\part[10] Een \texttt{umask} van 022
\end{parts}
\begin{tabular}[t]{llll}
  Wat & a & b & c\\
  \hline
  user - read & \hspace{2 cm} & \hspace{2 cm} & \hspace{2 cm}\\
  user - write & \hspace{2 cm} & \hspace{2 cm} & \hspace{2 cm}\\
  user - execute & \hspace{2 cm} & \hspace{2 cm} & \hspace{2 cm}\\
  group - read & \hspace{2 cm} & \hspace{2 cm} & \hspace{2 cm}\\
  group - write & \hspace{2 cm} & \hspace{2 cm} & \hspace{2 cm}\\
  group - execute & \hspace{2 cm} & \hspace{2 cm} & \hspace{2 cm}\\
  other - read & \hspace{2 cm} & \hspace{2 cm} & \hspace{2 cm}\\
  other - write & \hspace{2 cm} & \hspace{2 cm} & \hspace{2 cm}\\
  other - execute & \hspace{2 cm} & \hspace{2 cm} & \hspace{2 cm}\\
\end{tabular}

\begin{solution}
\begin{tabular}[t]{llll}
  Wat & a & b & c\\
  \hline
  user - read & y & y & y\\
  user - write & y & - & y\\
  user - execute & - & y & y\\
  group - read & y & - & y\\
  group - write & - & - & -\\
  group - execute & y & y & y\\
  other - read & - & - & y\\
  other - write & - & - & -\\
  other - execute & y & y & y\\
\end{tabular}
\end{solution}

\question[10] Installeer de officiele slackware Java Development Kit package. Geef de uitvoer van \texttt{javac -version}. Geef de gedane stappen
\begin{solution}
\begin{lstlisting}
wget ftp://ftp.nluug.nl/pub/os/Linux/distr/slackware/slackware-13.1/extra/jdk-6/jdk-6u20-i586-1.txz
root@slackbak:/home/kevin# installpkg jdk-6u20-i586-1.txz 
Verifying package jdk-6u20-i586-1.txz.
Installing package jdk-6u20-i586-1.txz:
PACKAGE DESCRIPTION:
# Java(TM) 2 Platform Standard Edition Development Kit 6.0 update 20.
#
# The Java 2 SDK software includes tools for developing, testing, and
# running programs written in the Java programming language.  This
# package contains everything you need to run Java(TM).
#
# For additional information, refer to this Sun Microsystems web page:
#   http://java.sun.com/
#
Executing install script for jdk-6u20-i586-1.txz.
Package jdk-6u20-i586-1.txz installed.
root@slackbak:/home/kevin# /usr/lib/java/bin/javac -version
javac 1.6.0_20
\end{lstlisting}
\end{solution}

\question[0] Welk commando kan voor de volgende taken gebruikt worden:
\begin{parts}
\part[2\half] Bekijk de inhoud van een tekstbestand.
\begin{solution}
\texttt{cat}, \texttt{less}, \ldots
\end{solution}

\part[2\half] Een programma moet op vaste tijden worden uitgevoerd. Hoe is dit in te stellen?
\begin{solution}
\texttt{cron}
\end{solution}

\part[2\half] Zoek naar de tekenreeks \emph{DMA} in de file \emph{/var/log/dmesg}
\begin{solution}
\texttt{grep DMA /var/log/dmesg}
\end{solution}

\part[2\half] Een map genaamd \emph{current} moet altijd verwijzen naar de meest recente versie van software, zonder de map te kopieren. Hoe is dit te realiseren?
\begin{solution}
\texttt{ln -s prog-x.y current}
\end{solution}

\part[2\half] Een standard output stream bevat de tekenreeks \emph{abcde123}, wat een wachtwoord is. Deze tekenreeks dient gefilterd te worden naar \emph{geheim}. Tip: dit kan met de simpele expressie \texttt{'s/abcde123/geheim/'}.
\begin{solution}
\texttt{ | sed -e 's/abcde123/geheim/'}
\end{solution}

\part[2\half] Volg realtime een logfile op \texttt{/var/log/syslog}, zodat iedere nieuwe entry op de stdout geprint word. 
\begin{solution}
\texttt{tail -f /var/log/syslog}
\end{solution}
\end{parts}
