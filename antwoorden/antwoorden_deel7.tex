% tirlnx01 - Materiaal om het keuzevak Linux te geven 
% op de Hogeschool Rotterdam.
% Copyright (C) 2010 - 2011  Paul Sohier, Kevin van der Vlist
%
% This program is free software: you can redistribute it and/or modify
% it under the terms of the GNU General Public License as published by
% the Free Software Foundation, either version 3 of the License, or
% (at your option) any later version.
%
% This program is distributed in the hope that it will be useful,
% but WITHOUT ANY WARRANTY; without even the implied warranty of
% MERCHANTABILITY or FITNESS FOR A PARTICULAR PURPOSE.  See the
% GNU General Public License for more details.
%
% You should have received a copy of the GNU General Public License
% along with this program.  If not, see <http://www.gnu.org/licenses/>.
%
% Kevin van der Vlist - kevin@kevinvandervlist.nl
% Paul Sohier - paul@paulsohier.nl

\section{Week 7}
Deze vragen gaan over de volgende hoofdstukken:
\begin{itemize}
\item[1.] Shell scripting
\end{itemize}

\question[10] Schrijf een script dat directories kan inlezen, en daarop volgordes kan sorteren (Bijvoorbeeld op grootte of alfabetische volgorde).
\begin{solution}
TODO
\end{solution}

\question[50] Maak met behulp van het case statement een menu dat weer bestaat uit verschillende opties, zoals:
    \begin{itemize}
      \item Een file in een variabele zetten.
      \item Een bestand kopieren
      \item Een bestand verplaatsen
      \item Een bestand aanpassen
      \item Een bestand emailen
      \item Een bestand verwijderen
    \end{itemize}
    Maak dit programma zo dat er om een wachtwoord gevraagd wordt bij het opstarten. Geeft de gebruiker een fout wachtwoord dan mag die persoon geen gebruik maken van het script. Zorg ervoor dat het wachtwoord wordt opgeslagen in een file (geheim.txt).
    Zorg ervoor dat alle menu opties ook werken (Eventueel met fake data, zoals bij email). Vraag om een bevestiging wanneer een actie niet ongedaan kan worden gemaakt. 

\textbf{Let op:} Doordat de mailserver niet is geconfigureerd zal je mailtje niet aankomen, het zal achter wel in de logfiles komen te staan. Je zal in de file \texttt{/var/log/maillog} dit zien:
\begin{lstlisting}
root@slackbak:/var/log# cat maillog
Mar 29 16:21:33 slackbak sendmail[2507]: p2TELXfV002507: from=paul, size=2524, class=0, nrcpts=1, msgid=<201103291421.p2TELXfV002507@slackbak.cmi-hro.nl>, relay=paul@localhost
Mar 29 16:21:33 slackbak sendmail[2507]: p2TELXfV002507: to=paul@hosthuis.nl, ctladdr=paul (1001/100), delay=00:00:00, xdelay=00:00:00, mailer=relay, pri=32524, relay=[127.0.0.1] [127.0.0.1], dsn=4.0.0, stat=Deferred: Connection refused by [127.0.0.1]
\end{lstlisting}

\begin{solution}
\begin{lstlisting}
#!/bin/bash

function menu
{
	echo "Kies een optie om uit te voeren: ";
	echo "1. Zet een bestand in een variable";
	echo "2. Kopieer een bestand naar een andere locatie";
	echo "3. Verplaats een bestand naar een andere locatie";
	echo "4. Pas een bestand aan via een editor";
	echo "5. Email een bestand naar een gebruiker";
	echo "6. Verwijder een bestand";
	
	echo "";
	
	echo "Voer het nummer in van wat je wilt, of q om af te sluiten";
}

function zetInVariable
{
	echo "Welke file wil je in een variable zetten?";
	read file;

	if [ -e $file ] ; then
		echo "We gaan nu ${file} in een variable zetten.";

#		$newData=`cat `;

		# Debug:
		echo $newData;
	else
		echo "${file} bestaat niet? ";
	fi

	sleep 3; # Even slapen zodat gebruiker rustig kan lezen
}

function kopieer
{
    echo "Welke file wil je kopieren?";
    read file
    echo "waarheen wil je hem kopieren?";
    read loc

    if [ -e $file ] ; then
	cp $file $loc
    else
	echo "${file} bestaat niet";
    fi
    sleep 3; # Even slapen;
}

function verplaats
{
    echo "welke file wil je verplaatsen?";

    read file
    echo "waarheen wil je hem verplaatsen?";
    read loc

    if [ -e $file ] ; then
	mv $file $loc
    else
	echo "${file} bestaat niet.";
    fi

    sleep 3;
}

function verwijder
{
    echo "Welke file wil je verwijderen?";

    read file

    echo "Weet je zeker dat je ${file} wilt verwijderen? Type J.";
    read ja

    if [ $ja == "J" ] ; then
	rm $file;
	echo "Verwijderd";
    fi
    sleep 3;
}

function edit
{
    echo "Welke file wil je aanpassen?";
    read file

    `emacs ${file}`;
}

function email
{
    echo "Welke file wil je sturen?";
    read file;

    echo "Waarheen wil je hem sturen?";
    read email;

    echo  "Welk onderwerp wil je hem geven?";
    read onder;

    if [ -e $file ]; then
	cat $file | /usr/bin/mail -s "$onder" "$email";
	echo "Verzonden";
    else
	echo "${file} bestaat niet.";
    fi
}

exit=0;
until [ $exit -gt 1 ]; do
      menu
      read optie
      echo "Gekozen voor ${optie}";
	case $optie in
	     "q") exit;;
	     "1") zetInVariable;;
	     "2") kopieer;;
	     "3") verplaats;;
	     "4") edit;;
	     "5") email;;
	     "6") verwijder;;

	esac
 done
\end{lstlisting}%$
\end{solution}

\question[20] Maak een script wat een gebruikersnaam of uid accepteert als parameter. Daarna zal het alle groepen printen waar deze gebruiker lid van is. 
\begin{solution}
\begin{lstlisting}
#!/bin/bash
function getNameFromUID {
    local LINE=`grep -E ".+:.+:$1:[0-9]+.+" /etc/passwd`
    for x in $LINE; do
        NAME=$x
        break
    done
}

function getGroupsFromName {
    for GROUP in `grep $1 /etc/group | perl -pe 's/([a-z]+):(.+)/\1/'`; do
        echo $GROUP
    done
}

if [ "$#" -eq "0" ]; then
    echo "Geef een gebruikersnaam of UID op."
    echo "./script.sh kevin"
    exit 1
fi

# Internal field seperator - bash shell built-in
OIFS=$IFS
IFS=':'

if [[ "$1" =~ ^[0-9]+$ ]]; then
    getNameFromUID $1
else
    NAME=$1
fi

echo "User $NAME zit in de volgende groepen:"
getGroupsFromName $NAME
IFS=$OIFS
exit 0
\end{lstlisting}
\end{solution}

\question[20] Maak een script wat backups maakt. Zoek zelf uit hoe. Enige eis is dat het script de directorie(s) (Er kunnen dus meerdere parameters opgegeven worden!) welke gebackuped moeten worden opgegeven kan worden als parameter. Bij het inleveren dien je uit te leggen waarom je welke keuzes hebt gemaakt.
\begin{solution}
\begin{lstlisting}
tar -cjf /backup/file.tar.bz2 /home/kevin/ /home/paul/
\end{lstlisting}
\end{solution}
