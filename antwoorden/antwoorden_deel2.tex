% tirlnx01 - Materiaal om het keuzevak Linux te geven 
% op de Hogeschool Rotterdam.
% Copyright (C) 2010 - 2011  Paul Sohier, Kevin van der Vlist
%
% This program is free software: you can redistribute it and/or modify
% it under the terms of the GNU General Public License as published by
% the Free Software Foundation, either version 3 of the License, or
% (at your option) any later version.
%
% This program is distributed in the hope that it will be useful,
% but WITHOUT ANY WARRANTY; without even the implied warranty of
% MERCHANTABILITY or FITNESS FOR A PARTICULAR PURPOSE.  See the
% GNU General Public License for more details.
%
% You should have received a copy of the GNU General Public License
% along with this program.  If not, see <http://www.gnu.org/licenses/>.
%
% Kevin van der Vlist - kevin@kevinvandervlist.nl
% Paul Sohier - paul@paulsohier.nl

\section{Week 2}
Deze vragen gaan over de volgende hoofdstukken:
\begin{itemize}
\item[1.] Het systeem
\item[2.] Inrichting
\item[3.] Editors
\end{itemize}

\question[10] In welke file zal het commando \texttt{halt} noteren dat het systeem gaat afsluiten? Wat is dit voor bestand? Geef de description
\begin{solution}
\begin{lstlisting}
kevin@slackbak:~$ man halt
/var/log/wtmp
kevin@slackbak:~$ man wtmp
The utmp file allows one to discover information about who is currently using the system. There may be more users currently using the system, because not all programs use utmp logging.
\end{lstlisting}
\end{solution}

\question[15] Bekijk (als root) de bestanden ``/etc/passwd'' en ``/etc/shadow''. Maak nu een nieuwe gebruiker aan op het systeem. Geef de veranderingen aan, en probeer te verklaren wat van iedere regel de velden betekenen. De velden zijn gescheiden door een ``:''.
\begin{solution}
\begin{lstlisting}
root@slackbak:/home/kevin# tail -n 1 /etc/passwd
test:x:1002:100:,,,:/home/test:/bin/bash
gebruiker:crypted pass:uid:gid:Full username:home dir:shell
root@slackbak:/home/kevin# tail -n 1 /etc/shadow
test:$1$kZ2D2/Mv$3zUicngYFa09RtoKiU/tn0:14988:0:99999:7:::
struct spwd {
      char          *sp_namp; /* user login name */
      char          *sp_pwdp; /* encrypted password */
      long int      sp_lstchg; /* last password change */
      long int      sp_min; /* days until change allowed. */
      long int      sp_max; /* days before change required */
      long int      sp_warn; /* days warning for expiration */
      long int      sp_inact; /* days before account inactive */
      long int      sp_expire; /* date when account expires */
      unsigned long int  sp_flag; /* reserved for future use */
}
\end{lstlisting}%$
\end{solution}

\question[5] Verwijder nu handmatig de toegevoegde gebruiker. Ruim ook de home-map op. Geef de gedane stappen.
\begin{solution}

Delete de regel in ``/etc/shadow'', delete regel in ``/etc/passwd''. Verwijder de home directory, en eventueel de groep in ``/etc/group''.
\end{solution}

\question[10] Wat gebeurt er bij de volgende wijziging?
\begin{lstlisting}
Origineel: 
kevin@slackbak:~$ grep initdefault /etc/inittab 
id:3:initdefault:
Gewijzigd:
kevin@slackbak:~$ grep initdefault /etc/inittab 
id:6:initdefault:
\end{lstlisting}
\begin{solution}

Dit word een reboot loop, door init level 6
\end{solution}

\question[0] Wij bieden een file system image aan:
Download ``image.ext2.tar.bz2'', te vinden in de map ``bestanden''. Pak dit bestand uit. Tip: downloaden kan met \texttt{wget}
\begin{solution}
\begin{lstlisting}
kevin@slackbak:~/disk$ dd if=/dev/zero bs=2048 count=5120 of=image.ext2
mkfs.ext2 -N 16 image.ext2
\end{lstlisting}%$
\end{solution}

\begin{parts}

\part[10] Wat is het voor bestand?
\begin{solution}
Ext2 file system image. 
\end{solution}

\part[10] Hoe is dit te mounten? Geef het commando. 
\begin{solution}
\begin{lstlisting}
mount image.ext2 /mount/location/ -o loop
\end{lstlisting}
\end{solution}

\part[10] Waarom werkt dit niet? Hoe kan dit gerepareerd worden? Geef het commando. 
\begin{solution}
\begin{lstlisting}
root@slackbak:/home/kevin/disk# mount image.ext2 /home/kevin/disk/mount/ -o loop
mount: you must specify the filesystem type
root@slackbak:/home/kevin/disk# mount image.ext2 -t ext2 /home/kevin/disk/mount/ -o loop
mount: wrong fs type, bad option, bad superblock on /dev/loop0,
       missing codepage or helper program, or other error
       In some cases useful info is found in syslog - try
       dmesg | tail  or so
iusaaset:/tmp# fsck.ext2 image.ext2
e2fsck 1.41.12 (17-May-2010)
fsck.ext2: Superblock invalid, trying backup blocks...
image.ext2 was not cleanly unmounted, check forced.
Pass 1: Checking inodes, blocks, and sizes
Pass 2: Checking directory structure
Pass 3: Checking directory connectivity
Pass 4: Checking reference counts
Pass 5: Checking group summary information
Free blocks count wrong for group #0 (8134, counted=8133).
Fix<y>? yes

Free blocks count wrong (10137, counted=10136).
Fix<y>? yes

Free inodes count wrong for group #1 (5, counted=4).
Fix<y>? yes

Free inodes count wrong (5, counted=4).
Fix<y>? yes


image.ext2: ***** FILE SYSTEM WAS MODIFIED *****
image.ext2: 12/16 files (0.0% non-contiguous), 104/10240 blocks
\end{lstlisting}
\end{solution}

\part[10] Voer het onderstaande commando uit in de gemounte ``image.ext2''. 
\begin{lstlisting}
kevin@slackbak:~/disk/mount$ for bestand in a b c d e; do touch $bestand; done
touch: cannot touch `e': No space left on device
\end{lstlisting}
Leg uit waarom er geen bestanden meer aangemaakt kunnen worden. 
\begin{solution}
Alle inodes zijn gebruikt:
\begin{lstlisting}
root@slackbak:/home/kevin/disk/mount# tune2fs -l /dev/loop0
[...]
Inode count:              16
[...]
Free inodes:              4
\end{lstlisting}
\end{solution}
\end{parts}

\question[20] Voer het onderstaande commando uit:
\begin{lstlisting}
kevin@slackbak:~/disk$ dd if=/dev/zero bs=2048 count=5120 of=image.ext3
\end{lstlisting}%$
Maak nu een ext3 filesystem aan op dit ``image.ext3'' bestand. Zorg ervoor dat 10 \% van de ruimte is gereserveerd voor \emph{root}. Mount daarna het bestand, en zet er een file op die ``hallo'' heet. Tar en bzip2 hem nu voor opslag. Geef de gebruikte commando's. 
\begin{solution}
\begin{lstlisting}
/sbin/mkfs -t ext3 -j -m 10 image.ext3
mount image.ext3 /dest/mount/point -o loop
touch /dest/mount/point/hallo
umount /dest/mount/point
tar -cjf image.ext3.tar.bz2 image.ext3
4 pt per deel
\end{lstlisting}
\end{solution}
