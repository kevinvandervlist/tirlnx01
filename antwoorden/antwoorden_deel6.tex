% tirlnx01 - Materiaal om het keuzevak Linux te geven 
% op de Hogeschool Rotterdam.
% Copyright (C) 2010 - 2011  Paul Sohier, Kevin van der Vlist
%
% This program is free software: you can redistribute it and/or modify
% it under the terms of the GNU General Public License as published by
% the Free Software Foundation, either version 3 of the License, or
% (at your option) any later version.
%
% This program is distributed in the hope that it will be useful,
% but WITHOUT ANY WARRANTY; without even the implied warranty of
% MERCHANTABILITY or FITNESS FOR A PARTICULAR PURPOSE.  See the
% GNU General Public License for more details.
%
% You should have received a copy of the GNU General Public License
% along with this program.  If not, see <http://www.gnu.org/licenses/>.
%
% Kevin van der Vlist - kevin@kevinvandervlist.nl
% Paul Sohier - paul@paulsohier.nl

\section{Week 6}
Deze vragen gaan over de volgende hoofdstukken:
\begin{itemize}
\item[1.] Shell scripting
\end{itemize}

\question[15] Maak een script dat de gebruiker twee keer om invoer vraagt, waarna deze met elkaar worden vergeleken.
\begin{solution}
\begin{lstlisting}
#!/bin/bash
echo ``Geef twee waarde om te vergelijken''
read antwoord1
read antwoord2

if [ $antwoord == $antwoord ]; then
    echo ``Antwoord1 is gelijk aan antwoord1''
else
    echo ``Antwoord1 is niet gelijk aan antwoord2''
fi
\end{lstlisting}
\end{solution}

\question[15] Maak een case/switch die een string van de input leest, en switcht op groen, geel, blauw en een default.
\begin{solution}
\begin{lstlisting}
#!/bin/bash
echo ``Geef twee waarde om te vergelijken''
read input
case "$input" in 
  "groen")
    echo "groen" ;;
  "geel")
    echo "geel" ;;
  "blauw")
    echo "blauw" ;;
  *)
    echo "default" ;;
esac
\end{lstlisting}%$
\end{solution}

\question[0] Geef een oplossing voor de volgende tests. Zie de eerste vraag als voorbeeld
\begin{parts}
\part[5] Test of een string een lengte van 0 heeft.
\begin{lstlisting}
[ -z "$string" ]
\end{lstlisting}%$

\part[5] Test of de lengte van een string 1 of groter is. 
\begin{solution}
\begin{lstlisting}
[ -n "$string" ]
\end{lstlisting}%$
\end{solution}

\part[5] Kijk of integer a groter is dan integer b.
\begin{solution}
\begin{lstlisting}
[ "$a" -gt "$b" ]
\end{lstlisting}%$
\end{solution}

\part[5] Kijk of een bestand een symbolic link is.
\begin{solution}
\begin{lstlisting}
[ -h "$bestand" ]
\end{lstlisting}%$
\end{solution}

\part[5] Kijk of een bestand schrijfbaar is en groter is dan 0
\begin{solution}
\begin{lstlisting}
[ -w "$bestand" ] && [ -s "$bestand" ]
\end{lstlisting}%$
\end{solution}

\part[5] Kijk of een bestand bestaat en een bestand is. Ook moet er gekeken worden of a kleiner of gelijk is als b
\begin{solution}
\begin{lstlisting}
# stat -c %s geeft filesize
[ -f "$1" ] && [ $(stat -c%s $1) -le $(stat -c%s $2) ]
\end{lstlisting}%$
\end{solution}
\end{parts}

\question[20] Maak een script wat een directory als parameter neemt, en van alle files print hoeveel hardlinks het heeft. Voorbeeld output:
\begin{lstlisting}
/home/kevin/files.gz: 2
/home/kevin/fixemacs.sh: 1
\end{lstlisting}
\begin{solution}
\begin{lstlisting}
dir=/home/kevin; for i in $dir/*; do echo -n "$dir$i: "; stat -c %h $i; done
\end{lstlisting}%$
\end{solution}

\question[20] Installeer lighttpd (versie 1.4.28). Zorg ervoor dat er openssl, bzip2 en zlib support meegecompileerd word. Een configuratiebestand kan gevonden worden in de map \texttt{lighttpd-1.4.28/tests/lighttpd.conf}. Zorg dat de webserver kan starten, en dat deze luistert op port 8080. Waarom is de webserver niet te bereiken vanaf andere computers dan localhost? Geef de gedane stappen.

Tip: Kijk voor meer informatie in appendix G, Source Installatie.
\begin{solution}
\begin{lstlisting}
./configure --prefix=/home/kevin/software/lighthttpd --with-bzip2 --with-zlib --with-openssl
make
make install
export SRCDIR=/tmp
/home/kevin/software/lighthttpd/sbin/lighttpd -f /home/kevin/software/lighthttpd/etc/lighttpd.conf
## bind to port (default: 80)
server.port                 = 2048
## bind to localhost (default: all interfaces)
server.bind                = "localhost"
\end{lstlisting}
\end{solution}
