% tirlnx01 - Materiaal om het keuzevak Linux te geven 
% op de Hogeschool Rotterdam.
% Copyright (C) 2010 - 2011  Paul Sohier, Kevin van der Vlist
%
% This program is free software: you can redistribute it and/or modify
% it under the terms of the GNU General Public License as published by
% the Free Software Foundation, either version 3 of the License, or
% (at your option) any later version.
%
% This program is distributed in the hope that it will be useful,
% but WITHOUT ANY WARRANTY; without even the implied warranty of
% MERCHANTABILITY or FITNESS FOR A PARTICULAR PURPOSE.  See the
% GNU General Public License for more details.
%
% You should have received a copy of the GNU General Public License
% along with this program.  If not, see <http://www.gnu.org/licenses/>.
%
% Kevin van der Vlist - kevin@kevinvandervlist.nl
% Paul Sohier - paul@paulsohier.nl

\documentclass{beamer}

\mode<presentation>

\usepackage[dutch]{babel}
\usepackage{listings}
%\usepackage{beamerthemesplit}

\lstset{ %
  language=bash,                % choose the language of the code
  basicstyle=\footnotesize,       % the size of the fonts that are used for the code
  numbers=left,                   % where to put the line-numbers
  numberstyle=\footnotesize,      % the size of the fonts that are used for the line-numbers
  numbersep=5pt,                  % how far the line-numbers are from the code
  showspaces=false,               % show spaces adding particular underscores
  showstringspaces=false,         % underline spaces within strings
  showtabs=false,                 % show tabs within strings adding particular underscores
  frame=lr,	                % adds left and right lines
  tabsize=2,	                % sets default tabsize to 2 spaces
  captionpos=b,                   % sets the caption-position to bottom
  breaklines=true,                % sets automatic line breaking
  breakatwhitespace=false,        % sets if automatic breaks should only happen at whitespace
%  escapeinside={\%*}{*)},         % if you want to add a comment within your code
  morekeywords={esac,fi,elseif,*,...}            % if you want to add more keywords to the set
}

\usetheme{Berlin}
\useinnertheme{rounded}
\usecolortheme{rose}
\setbeamertemplate{navigation symbols}{} 

\title{Keuzevak Linux - Week 6}
\author{Paul Sohier \and Kevin van der Vlist}
\institute{Versie $1.0$}
\date{\today}

\begin{document}

\begin{frame}
  \titlepage
\end{frame} 

\begin{frame}
  \frametitle{Inhoud}
  \tableofcontents
\end{frame}

\section{Wat en waarom}

\begin{frame}
  \frametitle{Shell scripting - wat en waarom}
  \begin{itemize}
  \item<1-> reeks opdrachten
  \item<2-> controle structuren
  \item<3-> automatiseren taken
  \item<4-> backups, gebruikers aanmaken, \ldots
  \end{itemize}
\end{frame}

\begin{frame}
  \frametitle{Shell scripting - begin}
  \begin{itemize}
  \item<1-> \texttt{shebang}: \texttt{\#!/bin/bash}
  \item<2-> opdrachten of controle structuren
  \item<3-> eindresultaat
  \end{itemize}
\end{frame}

\section{Variabelen}

\begin{frame}
  \frametitle{Shell scripting - variabelen}
  \begin{itemize}
  \item<1-> loosely typed
  \item<2-> refereren met \texttt{\$naam}%
  \end{itemize}
\end{frame}

\begin{frame}[fragile]
  \frametitle{Shell scripting - variabelen}
  \begin{lstlisting}
#!/bin/bash
var=2
var2=5
drie="drie"
echo $var2
echo "$var + $drie = $var2"
  \end{lstlisting}
  \begin{lstlisting}
3
2 + 3 = 5
  \end{lstlisting}
\end{frame}

\section{Systeemvariabelen}

\begin{frame}[fragile]
  \frametitle{Shell scripting - systeemvariabelen}
  \begin{itemize}
  \item<1-> ingesteld door omgeving
  \item<2-> export EDITOR=emacs
  \item<3-> simpel te gebruiken
  \end{itemize}
\end{frame}

\begin{frame}[fragile]
  \frametitle{Shell scripting - systeemvariabelen}
  \begin{itemize}
  \item \texttt{\$PATH}
  \item \texttt{\$TERM}
  \item \texttt{\$EDITOR}
  \item \texttt{\$DISPLAY}
  \item \ldots
  \end{itemize}
\end{frame}

\section{if}

\begin{frame}[fragile]
  \frametitle{Shell scripting - if}
  \begin{lstlisting}
if [ "Pasen" == "Pinksteren" ];
then
    echo "sint-juttemis"
fi
  \end{lstlisting}
\end{frame}

\section{if/else}

\begin{frame}[fragile]
  \frametitle{Shell scripting - if/else}
  \begin{lstlisting}
if [ "appel" == "banaan" ];
then
    echo "onbereikbaar"
else
    echo "appel en banaan zijn ongelijk"
fi
  \end{lstlisting}
\end{frame}

\section{if/elseif/else}

\begin{frame}[fragile]
  \frametitle{Shell scripting - if/elseif/else}
  \begin{lstlisting}
if [ "geluid" == "hard" ];
then
    echo "geluid is hard"
elseif [ "geluid" == "zacht" ];
then
    echo "geluid is zacht"
else
    echo "geluid is niet hard en niet zacht"
fi
  \end{lstlisting}
\end{frame}

\section{case/switch}

\begin{frame}[fragile]
  \frametitle{Shell scripting - case/switch}
  \begin{lstlisting}
functie="student"
case "$functie" in
    "student")
        echo "student" ;;
    "administratie")
        echo "administratie" ;;
    "docent")
        echo "docent" ;;
    *)
        echo "geen match" ;;
esac
  \end{lstlisting}%$
\end{frame}

\section{while}

\begin{frame}[fragile]
  \frametitle{Shell scripting - while}
  \begin{lstlisting}
i=0
while [ "$i" -lt "5" ]; do   
  let i++
  echo -n $i:
done
1:2:3:4:5:
  \end{lstlisting}
\end{frame}

\section{until}

\begin{frame}[fragile]
  \frametitle{Shell scripting - until}
  \begin{lstlisting}
i=10
until [ "$i" -lt "5" ]; do   
  let i--
  echo -n $i:
done
9:8:7:6:5:4:
  \end{lstlisting}
\end{frame}

\section{functies}

\begin{frame}[fragile]
  \frametitle{Shell scripting - functies}
  \begin{lstlisting}
function hallofunc {

}
  \end{lstlisting}
  \begin{lstlisting}
hallofunc() {

}
  \end{lstlisting}
\end{frame}

\section{functievariabelen}

\begin{frame}[fragile]
  \frametitle{Shell scripting - functievariabelen}
  \begin{lstlisting}
function hallofunc {
  echo hallo $1
}

hallofunc kevin
  \end{lstlisting}
  \begin{lstlisting}
hallofunc() {
  echo hallo $1
}
hallofunc kevin
  \end{lstlisting}
\end{frame}

\section{Expressies}

\begin{frame}[fragile]
  \frametitle{Shell scripting - string expressies}
  \begin{tabular}[t]{ll}
    Expressie & Waar als\ldots \\
    \hline
    -z string & de lengte van de string 0 is \\
    -n string & de lengte van string niet 0 is\\
    string1 == string2 & de twee strings gelijk zijn\\
    string1 != string2 & de twee strings ongelijk zijn\\
    string & de string niet NULL is\\
  \end{tabular}
  \begin{lstlisting}
if [ -z "$string" ]; then echo "lengte is 0"; fi
  \end{lstlisting}%$
\end{frame}

\begin{frame}[fragile]
  \frametitle{Shell scripting - integer expressies}
  \begin{tabular}[t]{ll}
    expressie & waar als\ldots \\
    \hline
    int1 -eq int2 & 1 == 2\\
    int1 -ne int2 & 1 != 2\\
    int1 -gt int2 & 1 $> $ 2\\
    int1 -ge int2 & 1 $\geq$ 2\\
    int1 -lt int2 & 1 $< $ 2\\
    int1 -le int2 & 1 $\leq$ 2\\
  \end{tabular}
  \begin{lstlisting}
if [ "1" -lt "2" ]; then echo "1 < 2"; fi
  \end{lstlisting}
\end{frame}

\begin{frame}[fragile]
  \frametitle{Shell scripting - file expressies}
  \begin{tabular}[t]{ll}
    Expressie & Waar als\ldots\\
    \hline
    -e file & file bestaat \\
    -r file & file bestaat en is readable \\
    -w file & file bestaat en is writeable \\
    -x file & file bestaat en is executable \\
    -f file & file bestaat en is een gewoon bestand \\
    -d file & file bestaat en is een directory\\
    -h file & file bestaat en is een symbolic link \\
    -s file & file bestaat en de grote is meer als 0 \\
  \end{tabular}
  \begin{lstlisting}
if [ -e "file" ]; then echo "file bestaat"; fi
  \end{lstlisting}%$
\end{frame}

\end{document}
