% tirlnx01 - Materiaal om het keuzevak Linux te geven 
% op de Hogeschool Rotterdam.
% Copyright (C) 2010 - 2011  Paul Sohier, Kevin van der Vlist
%
% This program is free software: you can redistribute it and/or modify
% it under the terms of the GNU General Public License as published by
% the Free Software Foundation, either version 3 of the License, or
% (at your option) any later version.
%
% This program is distributed in the hope that it will be useful,
% but WITHOUT ANY WARRANTY; without even the implied warranty of
% MERCHANTABILITY or FITNESS FOR A PARTICULAR PURPOSE.  See the
% GNU General Public License for more details.
%
% You should have received a copy of the GNU General Public License
% along with this program.  If not, see <http://www.gnu.org/licenses/>.
%
% Kevin van der Vlist - kevin@kevinvandervlist.nl
% Paul Sohier - paul@paulsohier.nl

\documentclass{beamer}

\mode<presentation>

\usepackage[dutch]{babel}
%\usepackage{beamerthemesplit}
\usetheme{Berlin}
\useinnertheme{rounded}
\usecolortheme{rose}
\setbeamertemplate{navigation symbols}{} 

\title{Keuzevak Linux - Week 1}
\author{Paul Sohier \and Kevin van der Vlist}
\institute{Versie $1.0$}
\date{\today}

\begin{document}

\section{Welkom}
\frame{
  \titlepage
} 

\frame{
  \frametitle{Inhoud}
  \tableofcontents
}

\section{Informatie}
\frame{
  \frametitle{Informatie en benodigdheden}
  \begin{itemize}
    \item<1-> dictaat, presentaties, vragen te vinden op: \url{http://www.hosthuis.nl/linux/}
    \item<1-> geen tentamen, enkel vragen inleveren voor cijfer
    \item<1-> huiswerk inleveren op \url{huiswerk@paulsohier.nl}
    \item<1-> vermeld je naam en student nummer
    \item<1-> deadline: vrijdag van de laatste lesweek
  \end{itemize}
}

\frame{
  \frametitle{Dictaat printen - geen verplichting}
  \begin{itemize}
    \item<1-> kosten printen dictaat en inbinden (Op school): $10,92 + 3,50 = 14,48$ euro
    \item<1-> bestellen bij ons: 10 euro
    \item<1-> keuze enkel of dubbelzijdig
  \end{itemize}
}

\frame{
  \frametitle{Voorkennis}
  \begin{itemize}
    \item<1-> basis kennis computers
    \item<1-> basis programeerkennis
    \item<1-> interesse in \emph{Linux}
  \end{itemize}
}

\section{Wat is Linux}
\frame{
  \frametitle{Wat is Linux - Geschiedenis}
  \begin{itemize}
    \item<1-> Finse student begon in 1991
    \item<1-> eerste versie niet bruikbaar, hackers speeltje
    \item<1-> gebruik makend van GNU userland
  \end{itemize}
}

\frame{
  \frametitle{Wat is Linux - De kracht}
  \begin{itemize}
    \item open, iedereen kan aanpassen
    \item zo te gebruiken, voor iedereen
  \end{itemize}
}

\frame
{
  \frametitle{Wat is Linux - De problemen}
  \begin{itemize}
    \item open, iedereen kan zo in de source kijken en problemen vinden
    \item veel (verschillende) versies
  \end{itemize}
}

\section{Linux en Hardware}
\frame{
  \frametitle{Linux en Hardware}
  \begin{itemize}
    \item harde schijven heten /dev/sdX
    \item alles via libata heeft zelfde naam
    \item vroeger /dev/hdX
  \end{itemize}
}

\section{Installatie}
\frame{
  \frametitle{Installatie - Slackware}
  \begin{itemize}
    \item<1-> Slackware, ``ouderwets''
    \item<1-> geen GUI
    \item<1-> gebruik virtualbox voor installatie
  \end{itemize}
}

\frame{
 \frametitle{Installatie - Grub}
 \begin{itemize}
   \item<1-> vervanging van LiLo
   \item<1-> LiLo is ouderwets
   \item<1-> meer mogelijkheden met Grub
 \end{itemize}
}

\end{document}
