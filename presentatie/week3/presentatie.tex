% tirlnx01 - Materiaal om het keuzevak Linux te geven 
% op de Hogeschool Rotterdam.
% Copyright (C) 2010 - 2011  Paul Sohier, Kevin van der Vlist
%
% This program is free software: you can redistribute it and/or modify
% it under the terms of the GNU General Public License as published by
% the Free Software Foundation, either version 3 of the License, or
% (at your option) any later version.
%
% This program is distributed in the hope that it will be useful,
% but WITHOUT ANY WARRANTY; without even the implied warranty of
% MERCHANTABILITY or FITNESS FOR A PARTICULAR PURPOSE.  See the
% GNU General Public License for more details.
%
% You should have received a copy of the GNU General Public License
% along with this program.  If not, see <http://www.gnu.org/licenses/>.
%
% Kevin van der Vlist - kevin@kevinvandervlist.nl
% Paul Sohier - paul@paulsohier.nl

\documentclass{beamer}

\mode<presentation>

\usepackage[dutch]{babel}
%\usepackage{beamerthemesplit}

\usepackage{listings}
%\usepackage{beamerthemesplit}

\lstset{ %
  language=bash,                % choose the language of the code
  basicstyle=\footnotesize,       % the size of the fonts that are used for the code
  numbers=left,                   % where to put the line-numbers
  numberstyle=\footnotesize,      % the size of the fonts that are used for the line-numbers
  numbersep=5pt,                  % how far the line-numbers are from the code
  showspaces=false,               % show spaces adding particular underscores
  showstringspaces=false,         % underline spaces within strings
  showtabs=false,                 % show tabs within strings adding particular underscores
  frame=lr,	                % adds left and right lines
  tabsize=2,	                % sets default tabsize to 2 spaces
  captionpos=b,                   % sets the caption-position to bottom
  breaklines=true,                % sets automatic line breaking
  breakatwhitespace=false,        % sets if automatic breaks should only happen at whitespace
%  escapeinside={\%*}{*)},         % if you want to add a comment within your code
  morekeywords={*,...}            % if you want to add more keywords to the set
}


\usetheme{Berlin}
\useinnertheme{rounded}
\usecolortheme{rose}
\setbeamertemplate{navigation symbols}{} 

\title{Keuzevak Linux - Week 3}
\author{Paul Sohier \and Kevin van der Vlist}
\institute{Versie $1.0$}
\date{\today}

\begin{document}

\begin{frame}
  \titlepage
\end{frame} 

\begin{frame}
  \frametitle{Inhoud}
  \tableofcontents
\end{frame}

\section{User management}
\begin{frame}
  \frametitle{User management - Commando's}
  \begin{itemize}
    \item<1-> adduser
    \item<2-> who
    \item<2-> w
    \item<3-> last
    \item<4-> usermod
    \item<5-> chown
  \end{itemize}
\end{frame}

\begin{frame}
  \frametitle{User management - User Identifiers}
  \begin{itemize}
    \item<1-> iedere user unieke ID
    \item<2-> gebruikt voor rechten, service, bestands eigenaren
  \end{itemize}
\end{frame}

\begin{frame}
  \frametitle{User management - Real vs effective user id}
  \begin{itemize}
    \item<1-> real user id: echte user id
    \item<2-> effective user id: meestal real user id, soms niet
    \item<3-> effective user id, het UID waar het proces echt onder draait
    \item<4-> voorbeeld, zie dictaat blz. 53
  \end{itemize}
\end{frame}

\begin{frame}
  \frametitle{User management - File mode bits}
  \begin{itemize}
    \item<1-> gebruikt om rechten op te slaan
    \item<2-> te zien met \texttt{ls -l map}
  \end{itemize}
\end{frame}

\begin{frame}[fragile]
  \frametitle{User management - File mode bits}
  \begin{lstlisting}
kevin@slackbak:~$ ls -l /home/                                                                                                                                    
total 28                                                                                                                                                              
drwxr-xr-x 2 root  root   4096 Feb 13  2010 ftp                                                                                                                       
drwxr-x--x 3 kevin users  4096 Dec  8 09:39 kevin                                                                                                                     
drwx------ 2 root  root  16384 Dec  6 13:25 lost+found                                                                                                                
drwx--x--x 2 paul  users  4096 Dec  8 09:10 paul                                                                                                                      
-rw-r--r-- 1 kevin users 0 Dec  8 09:58 bestand        
\end{lstlisting}%$
\begin{itemize}
  \item<2-> eerste kolom defineert rechten
  \item<3-> specifieke betekenis beschreven in dictaat: pagina 52
\end{itemize}
\end{frame}

\section{Process management}
\begin{frame}
  \frametitle{Process management - Commando's}
  \begin{itemize}
    \item<1-> top
    \item<2-> ps
    \item<3-> kill
  \end{itemize}
\end{frame}

\begin{frame}
  \frametitle{Process management - PID, UID PPID in ps}
  \begin{itemize}
    \item<1-> PID: process ID
    \item<2-> UID: effective UID het process onder draait
    \item<3-> PPID: parent process id
    \item<4-> Zie voor verdere uitleg, dictaat bladzijde 57
  \end{itemize}
\end{frame}

\begin{frame}
  \frametitle{Process management - Terminations}
  \begin{itemize}
    \item<1-> process be\"{e}ndigen met \texttt{kill}
    \item<2-> \texttt gebruikt signals
    \item<3-> zombies killen kan enkel via parent
  \end{itemize}
\end{frame}

\end{document}
