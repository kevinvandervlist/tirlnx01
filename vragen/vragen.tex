% tirlnx01 - Materiaal om het keuzevak Linux te geven 
% op de Hogeschool Rotterdam.
% Copyright (C) 2010 - 2011  Paul Sohier, Kevin van der Vlist
%
% This program is free software: you can redistribute it and/or modify
% it under the terms of the GNU General Public License as published by
% the Free Software Foundation, either version 3 of the License, or
% (at your option) any later version.
%
% This program is distributed in the hope that it will be useful,
% but WITHOUT ANY WARRANTY; without even the implied warranty of
% MERCHANTABILITY or FITNESS FOR A PARTICULAR PURPOSE.  See the
% GNU General Public License for more details.
%
% You should have received a copy of the GNU General Public License
% along with this program.  If not, see <http://www.gnu.org/licenses/>.
%
% Kevin van der Vlist - kevin@kevinvandervlist.nl
% Paul Sohier - paul@paulsohier.nl

\documentclass[a4paper,11pt]{exam}
% Hier hebben we de preamble, alle document settings moeten hier:
\usepackage{graphicx}
\usepackage{url}
\usepackage{appendix}
\usepackage[titles]{tocloft}
\usepackage[dutch]{babel}
\usepackage{listings}
\usepackage{makeidx}
\usepackage{float}
\usepackage[hypertexnames=false]{hyperref}

% Custom LaTeX commands:
% ~ == {\raise.17ex\hbox{$\scriptstyle\sim$}}
\newcommand{\customtilde}{\raise.17ex\hbox{$\scriptstyle\sim$}}
% Meta info pdf:
\hypersetup{
%bookmarks=true, % show bookmarks bar?
unicode=false, % non-Latin characters in Acrobat’s bookmarks
pdftoolbar=true, % show Acrobat’s toolbar?
pdfmenubar=false, % show Acrobat’s menu?
pdffitwindow=false, % window fit to page when opened
%pdfstartview={FitH}, % fits the width of the page to the window
pdftitle={TIRLNX01 Vragen}, % title
pdfauthor={Paul Sohier, Kevin van der Vlist}, % author
pdfsubject={Inleiding Linux}, % subject of the document
pdfcreator={make}, % creator of the document
pdfproducer={make}, % producer of the document
pdfkeywords={Linux} {Basis}, % list of keywords
pdfnewwindow=true, % links in new window
colorlinks=false, % false: boxed links; true: colored links
linkcolor=black, % color of internal links
citecolor=green, % color of links to bibliography
filecolor=magenta, % color of file links
urlcolor=cyan % color of external links
}

% Paragrafen hebben een witregel ertussen, en geen indent tab:
\setlength{\parindent}{0.0in}
\setlength{\parskip}{0.1in}

% End of title + blz.
% Table of content depth van 4, dus tm paragraph
\setcounter{tocdepth}{4}
%\renewcommand{\baselinestretch}{1.5} 1.5 regelafstand 

%Pas listings aan zodat ze duidelijker zijn
\lstset{ %
  language=bash,                % choose the language of the code
  basicstyle=\footnotesize,       % the size of the fonts that are used for the code
  numbers=left,                   % where to put the line-numbers
  numberstyle=\footnotesize,      % the size of the fonts that are used for the line-numbers
  numbersep=5pt,                  % how far the line-numbers are from the code
  showspaces=false,               % show spaces adding particular underscores
  showstringspaces=false,         % underline spaces within strings
  showtabs=false,                 % show tabs within strings adding particular underscores
  frame=lr,	                % adds left and right lines
  tabsize=2,	                % sets default tabsize to 2 spaces
  captionpos=b,                   % sets the caption-position to bottom
  breaklines=true,                % sets automatic line breaking
  breakatwhitespace=false,        % sets if automatic breaks should only happen at whitespace
%  escapeinside={\%*}{*)},         % if you want to add a comment within your code
  morekeywords={*,...}            % if you want to add more keywords to the set
}
%hyperref aanpassingen
\hypersetup{pdfborder=0 0 0}

% Onderstaande is voor de dots tussen chapter title + blz. 
\makeatletter
\renewcommand*\l@section[2]{%
  \ifnum \c@tocdepth >\m@ne
    \addpenalty{-\@highpenalty}%
    \vskip 1.0em \@plus\p@
    \setlength\@tempdima{1.5em}%
    \begingroup
      \parindent \z@ \rightskip \@pnumwidth
      \parfillskip -\@pnumwidth
      \leavevmode \bfseries
      \advance\leftskip\@tempdima
      \hskip -\leftskip
      #1\nobreak\normalfont\leaders\hbox{$\m@th
        \mkern \@dotsep mu\hbox{.}\mkern \@dotsep
        mu$}\hfill\nobreak\hb@xt@\@pnumwidth{\hss #2}\par
      \penalty\@highpenalty
    \endgroup
  \fi}
\makeatother

% En punten
\addpoints
% In brackets in margin
\pointsinmargin
\boxedpoints

\framedsolutions % defines the style of the solution environment

\renewcommand{\solutiontitle}{\noindent\textbf{Antwoord: }\par\noindent}

\runningfooter{Paul Sohier\\Kevin van der Vlist}{TIRLNX01\\Hogeschool Rotterdam}{\thepage}


% Einde preamble, begin document; 
\begin{document}


% Title page
\begin{titlepage}
\vspace*{\fill}
\begin{center}

\textsc{\Huge \bfseries TIRLNX01}\\[0.5cm]
{\large \bfseries Keuzevak Linux}\\[0.25cm]
{\large \bfseries Vragen}\\[1.5cm]


\begin{minipage}{0.4\textwidth}
\begin{flushleft} \large
\texttt{Auteurs:}\\
\ \\
\textsc{Kevin van der Vlist}\\
\href{mailto:kevin@kevinvandervlist.nl}{kevin@kevinvandervlist.nl}\\
\ \\
\texttt{en}\\
\ \\
\textsc{Paul Sohier}\\
\href{mailto:paul@paulsohier.nl}{paul@paulsohier.nl}
\end{flushleft}
\end{minipage}
\begin{minipage}{0.5\textwidth}

\begin{flushright} \large
\texttt{Gedoceerd te:}\\
\ \\
\textsc{Hogeschool Rotterdam}\\
Vestiging Academieplein
\end{flushright}
\end{minipage}
{\ }\\[1.5cm]
{\large Versie 1.1}\\[0.75cm]
{\large \today}
\vfill
\end{center}
\vspace*{\fill}
\end{titlepage}

% Geen pagina no 1 op title
\thispagestyle{empty}

\clearpage

\vspace*{\fill}
\begin{abstract}
Om een cijfer te halen voor dit keuzevak dien je antwoorden te geven op alle vragen in dit dictaat. 
Omdat er geen tentamen word gegeven, zullen al deze vragen op papier moeten worden ingeleverd bij de begeleidend docent. 
Na het inleveren zullen deze opgaven worden nagekeken, waarna er een cijfer word bepaald aan de hand van het aantal behaalde punten. 

Het maximaal te behalen punten staat per vraag aangegeven.
De berekening die resulteert in het eindcijfer word als volgt gedaan:
$$
\frac{behaalde\_punten}{\numpoints} \cdot 100 = eindcijfer
$$

Mochten sommige (delen van) vragen te moeilijk zijn is het dus aan te raden om deze in ieder geval tijdelijk over te slaan.
Bekijk daarna of de punten ook nodig zijn. Succes.
\end{abstract}
\vspace*{\fill}

\clearpage

% De table of contents + toc in toc:
\tableofcontents
%\addcontentsline{toc}{chapter}{\numberline{}Inhoudsopgave}

% Nu kunnen we de losse hoofdstukken gaan includen. 
% Includen gebeurt met basename, dus zonder .tex

% Questions worden in deel 1 geopend
% tirlnx01 - Materiaal om het keuzevak Linux te geven 
% op de Hogeschool Rotterdam.
% Copyright (C) 2010 - 2011  Paul Sohier, Kevin van der Vlist
%
% This program is free software: you can redistribute it and/or modify
% it under the terms of the GNU General Public License as published by
% the Free Software Foundation, either version 3 of the License, or
% (at your option) any later version.
%
% This program is distributed in the hope that it will be useful,
% but WITHOUT ANY WARRANTY; without even the implied warranty of
% MERCHANTABILITY or FITNESS FOR A PARTICULAR PURPOSE.  See the
% GNU General Public License for more details.
%
% You should have received a copy of the GNU General Public License
% along with this program.  If not, see <http://www.gnu.org/licenses/>.
%
% Kevin van der Vlist - kevin@kevinvandervlist.nl
% Paul Sohier - paul@paulsohier.nl

\section{Week 1}

Deze vragen gaan over de volgende hoofdstukken:
\begin{itemize}
\item[1.] Introductie
\item[2.] Linux en Hardware
\item[3.] Installatie
\end{itemize}

\begin{questions}

\question[0] Installeer slackware
\question[50] Wat zijn de namen van de eerste partitie van de eerste schijf, en van de derde partitie van de tweede schijf? Ga er vanuit dat er gebruik wordt gemaakt van libata. 
\begin{solution}
/dev/sda1\\
/dev/sdb3
\end{solution}
\question[50] Maak een rescue optie voor GRUB zodat er geboot kan worden met \texttt{bash} als \emph{init daemon}. Geef het bestand + de aanpassingen die gemaakt zijn. \footnote{In sommige gevallen lijkt het erop dat de rescue mode niet meer werkt en is vastgelopen, dit is echter niet het geval, de rescue mode is al helemaal opgestart, maar daarna is nog een melding op STOUT geplaatst. Als je op enter drukt heb je vaak een werkende rescue mode.} 
\begin{solution}
\begin{lstlisting}
kevin@slackbak:~$ cat /boot/grub/menu.lst
  title Linux on (/dev/sda1)
  root (hd0,0)
  kernel /boot/vmlinuz root=/dev/sda1 ro vga=normal init=/bin/bash
\end{lstlisting}%$
\end{solution}

% tirlnx01 - Materiaal om het keuzevak Linux te geven 
% op de Hogeschool Rotterdam.
% Copyright (C) 2010 - 2011  Paul Sohier, Kevin van der Vlist
%
% This program is free software: you can redistribute it and/or modify
% it under the terms of the GNU General Public License as published by
% the Free Software Foundation, either version 3 of the License, or
% (at your option) any later version.
%
% This program is distributed in the hope that it will be useful,
% but WITHOUT ANY WARRANTY; without even the implied warranty of
% MERCHANTABILITY or FITNESS FOR A PARTICULAR PURPOSE.  See the
% GNU General Public License for more details.
%
% You should have received a copy of the GNU General Public License
% along with this program.  If not, see <http://www.gnu.org/licenses/>.
%
% Kevin van der Vlist - kevin@kevinvandervlist.nl
% Paul Sohier - paul@paulsohier.nl

\section{Week 2}
Deze vragen gaan over de volgende hoofdstukken:
\begin{itemize}
\item[1.] Het systeem
\item[2.] Inrichting
\item[3.] Editors
\end{itemize}

\question[10] In welke file zal het commando \texttt{halt} noteren dat het systeem gaat afsluiten? Wat is dit voor bestand? Geef de description
\begin{solution}
\begin{lstlisting}
kevin@slackbak:~$ man halt
/var/log/wtmp
kevin@slackbak:~$ man wtmp
The utmp file allows one to discover information about who is currently using the system. There may be more users currently using the system, because not all programs use utmp logging.
\end{lstlisting}
\end{solution}

\question[15] Bekijk (als root) de bestanden ``/etc/passwd'' en ``/etc/shadow''. Maak nu een nieuwe gebruiker aan op het systeem. Geef de veranderingen aan, en probeer te verklaren wat van iedere regel de velden betekenen. De velden zijn gescheiden door een ``:''.
\begin{solution}
\begin{lstlisting}
root@slackbak:/home/kevin# tail -n 1 /etc/passwd
test:x:1002:100:,,,:/home/test:/bin/bash
gebruiker:crypted pass:uid:gid:Full username:home dir:shell
root@slackbak:/home/kevin# tail -n 1 /etc/shadow
test:$1$kZ2D2/Mv$3zUicngYFa09RtoKiU/tn0:14988:0:99999:7:::
struct spwd {
      char          *sp_namp; /* user login name */
      char          *sp_pwdp; /* encrypted password */
      long int      sp_lstchg; /* last password change */
      long int      sp_min; /* days until change allowed. */
      long int      sp_max; /* days before change required */
      long int      sp_warn; /* days warning for expiration */
      long int      sp_inact; /* days before account inactive */
      long int      sp_expire; /* date when account expires */
      unsigned long int  sp_flag; /* reserved for future use */
}
\end{lstlisting}%$
\end{solution}

\question[5] Verwijder nu handmatig de toegevoegde gebruiker. Ruim ook de home-map op. Geef de gedane stappen.
\begin{solution}

Delete de regel in ``/etc/shadow'', delete regel in ``/etc/passwd''. Verwijder de home directory, en eventueel de groep in ``/etc/group''.
\end{solution}

\question[10] Wat gebeurt er bij de volgende wijziging?
\begin{lstlisting}
Origineel: 
kevin@slackbak:~$ grep initdefault /etc/inittab 
id:3:initdefault:
Gewijzigd:
kevin@slackbak:~$ grep initdefault /etc/inittab 
id:6:initdefault:
\end{lstlisting}
\begin{solution}

Dit word een reboot loop, door init level 6
\end{solution}

\question[0] Wij bieden een file system image aan:
Download ``image.ext2.tar.bz2'', te vinden in de map ``bestanden''. Pak dit bestand uit. Tip: downloaden kan met \texttt{wget}
\begin{solution}
\begin{lstlisting}
kevin@slackbak:~/disk$ dd if=/dev/zero bs=2048 count=5120 of=image.ext2
mkfs.ext2 -N 16 image.ext2
\end{lstlisting}%$
\end{solution}

\begin{parts}

\part[10] Wat is het voor bestand?
\begin{solution}
Ext2 file system image. 
\end{solution}

\part[10] Hoe is dit te mounten? Geef het commando. 
\begin{solution}
\begin{lstlisting}
mount image.ext2 /mount/location/ -o loop
\end{lstlisting}
\end{solution}

\part[10] Waarom werkt dit niet? Hoe kan dit gerepareerd worden? Geef het commando. 
\begin{solution}
\begin{lstlisting}
root@slackbak:/home/kevin/disk# mount image.ext2 /home/kevin/disk/mount/ -o loop
mount: you must specify the filesystem type
root@slackbak:/home/kevin/disk# mount image.ext2 -t ext2 /home/kevin/disk/mount/ -o loop
mount: wrong fs type, bad option, bad superblock on /dev/loop0,
       missing codepage or helper program, or other error
       In some cases useful info is found in syslog - try
       dmesg | tail  or so
iusaaset:/tmp# fsck.ext2 image.ext2
e2fsck 1.41.12 (17-May-2010)
fsck.ext2: Superblock invalid, trying backup blocks...
image.ext2 was not cleanly unmounted, check forced.
Pass 1: Checking inodes, blocks, and sizes
Pass 2: Checking directory structure
Pass 3: Checking directory connectivity
Pass 4: Checking reference counts
Pass 5: Checking group summary information
Free blocks count wrong for group #0 (8134, counted=8133).
Fix<y>? yes

Free blocks count wrong (10137, counted=10136).
Fix<y>? yes

Free inodes count wrong for group #1 (5, counted=4).
Fix<y>? yes

Free inodes count wrong (5, counted=4).
Fix<y>? yes


image.ext2: ***** FILE SYSTEM WAS MODIFIED *****
image.ext2: 12/16 files (0.0% non-contiguous), 104/10240 blocks
\end{lstlisting}
\end{solution}

\part[10] Voer het onderstaande commando uit in de gemounte ``image.ext2''. 
\begin{lstlisting}
kevin@slackbak:~/disk/mount$ for bestand in a b c d e; do touch $bestand; done
touch: cannot touch `e': No space left on device
\end{lstlisting}
Leg uit waarom er geen bestanden meer aangemaakt kunnen worden. 
\begin{solution}
Alle inodes zijn gebruikt:
\begin{lstlisting}
root@slackbak:/home/kevin/disk/mount# tune2fs -l /dev/loop0
[...]
Inode count:              16
[...]
Free inodes:              4
\end{lstlisting}
\end{solution}
\end{parts}

\question[20] Voer het onderstaande commando uit:
\begin{lstlisting}
kevin@slackbak:~/disk$ dd if=/dev/zero bs=2048 count=5120 of=image.ext3
\end{lstlisting}%$
Maak nu een ext3 filesystem aan op dit ``image.ext3'' bestand. Zorg ervoor dat 10 \% van de ruimte is gereserveerd voor \emph{root}. Mount daarna het bestand, en zet er een file op die ``hallo'' heet. Tar en bzip2 hem nu voor opslag. Geef de gebruikte commando's. 
\begin{solution}
\begin{lstlisting}
/sbin/mkfs -t ext3 -j -m 10 image.ext3
mount image.ext3 /dest/mount/point -o loop
touch /dest/mount/point/hallo
umount /dest/mount/point
tar -cjf image.ext3.tar.bz2 image.ext3
4 pt per deel
\end{lstlisting}
\end{solution}

% tirlnx01 - Materiaal om het keuzevak Linux te geven 
% op de Hogeschool Rotterdam.
% Copyright (C) 2010 - 2011  Paul Sohier, Kevin van der Vlist
%
% This program is free software: you can redistribute it and/or modify
% it under the terms of the GNU General Public License as published by
% the Free Software Foundation, either version 3 of the License, or
% (at your option) any later version.
%
% This program is distributed in the hope that it will be useful,
% but WITHOUT ANY WARRANTY; without even the implied warranty of
% MERCHANTABILITY or FITNESS FOR A PARTICULAR PURPOSE.  See the
% GNU General Public License for more details.
%
% You should have received a copy of the GNU General Public License
% along with this program.  If not, see <http://www.gnu.org/licenses/>.
%
% Kevin van der Vlist - kevin@kevinvandervlist.nl
% Paul Sohier - paul@paulsohier.nl

\section{Week 3}
Deze vragen gaan over de volgende hoofdstukken:
\begin{itemize}
\item[1.] User management
\item[2.] Proces management
\end{itemize}

\question[0] Kijk naar de volgende shell log:
\begin{lstlisting}
kevin@slackbak:~$ getent passwd root
root:x:0:0::/root:/bin/bash
kevin@slackbak:~$ getent passwd paul
paul:x:1001:100:,,,:/home/paul:/bin/bash
kevin@slackbak:~$ getent passwd kevin
kevin:x:1000:100:,,,:/home/kevin:/bin/bash
kevin@slackbak:~$ whoami
kevin
kevin@slackbak:~$ cp /usr/bin/top /tmp/top 
kevin@slackbak:~$ chmod 755 /tmp/top
kevin@slackbak:~$ chmod u+s /tmp/top
\end{lstlisting}%$

\begin{parts}
\part[15] Wat zijn nu de \emph{effective uid} en de \emph{real uid} van het volgende? Verklaar je antwoord.
\begin{lstlisting}
paul@slackbak:~$ /tmp/top 
\end{lstlisting}%$

\begin{solution}
\begin{lstlisting}
uid:  1001
euid: 1000
gid:  100
egid: 100
UID verhaal, sticky bit
\end{lstlisting}%$
\end{solution}

\part[15] Het proces van het programma \texttt{top} blijft actief. Wat gebeurt er nu met het \emph{effective uid} en het \emph{real uid} na de onderstaande stappen? En van nieuwe processen op \texttt{/tmp/top}?
\begin{lstlisting}
root@slackbak:~$ chown paul /tmp/top 
root@slackbak:~$ chmod u+s /tmp/top
\end{lstlisting}
\begin{solution}
\begin{lstlisting}
Actieve proces veranderd niet, zit al in memory. Nieuwe processen:
uid:  uid user
euid: 1001 (paul)
gid:  gid user
egid: egid user
\end{lstlisting}
\end{solution}
\end{parts}

\question[0] Bekijk de volgende situatie:
\begin{lstlisting}
kevin@slackbak:/tmp/map$ ls -lhR
.:
total 4.0K
drwxr--r-x 2 kevin users 4.0K 2011-01-16 13:28 map/

./map:
total 0
--w-r--r-- 1 kevin users 0 2011-01-16 13:28 file
\end{lstlisting}%$
\begin{parts}
\part[5] Mag de gebruiker kevin de map ``map'' in? En mag die het bestand ``file'' uitlezen?
\begin{solution}
map: ja, x-bit\\
file: ja, world
\end{solution}

\part[5] Mag een lid van de groep users de map ``map'' in? En mag die het bestand ``file'' uitlezen?
\begin{solution}
map: world x-bit, ja\\
file: ja, world
\end{solution}

\part[5] Mag een ander, dus ``other'', de map ``map'' in? En mag die het bestand ``file'' uitlezen?
\begin{solution}
map: ja, x-bit\\
file: ja, r-bit
\end{solution}
\end{parts}

\question[10] Bekijk welk \emph{pid} bij de actieve \texttt{bash} shell hoort. Kijk ook wat de parent is. Geef deze \emph{pid's} en de \emph{CMD's} van deze processen.
\begin{solution}
\begin{lstlisting}
kevin@slackbak:~$ ps -ef | grep bash
kevin     4413  4412  1 12:54 pts/0    00:00:00 -bash
kevin     4427  4413  0 12:54 pts/0    00:00:00 grep bash
kevin@slackbak:~$ ps -ef | grep 4412
kevin     4412  4409  0 12:54 ?        00:00:00 sshd: kevin@pts/0
kevin     4413  4412  0 12:54 pts/0    00:00:00 -bash
kevin     4448  4413  0 12:54 pts/0    00:00:00 grep 4451
# Note: ppid via console is 1
\end{lstlisting}
\end{solution}

\question[15] Stuur een \emph{SIGTERM} naar de \texttt{bash} shell. Wat gebeurt er? Stuur ook een \emph{SIGKILL} naar de \texttt{bash} shell, wat gebeurt er dan? Verklaar je antwoord.
\begin{solution}
\begin{lstlisting}
kevin@slackbak:~$ kill -15 4413
kevin@slackbak:~$ ps -ef | grep bash
kevin     4413  4412  0 12:54 pts/0    00:00:00 -bash
kevin     4448  4413  0 12:55 pts/0    00:00:00 grep bash
# SIGTERM wordt afgevangen - gebeurt niets
kevin@slackbak:~$ kill -9 4413
Connection to slack.icyx.nl closed
# Dus: bash sluit, waarna sshd zichzelf 'netjes' afsluit.
kevin@slackbak:~$ kill -15 4451Connection to slack.icyx.nl closed by remote host.
Connection to slack.icyx.nl closed.
# SIGKILL kan niet worden afgevangen, nu zal het proces aan de andere kant dood zijn. Sessie gaat dus kapot.
# Note: Op een console word login actief
\end{lstlisting}
\end{solution}

\question[0] Zombies:
\begin{parts}
\part[10] Wat is een zombie proces?
\begin{solution}
Een parent proces bekijkt de exit status van een child niet. Dit child proces blijft wachten met deze status, terwijl de parent al dood is. 
\end{solution}

\part[10] Waarom kan je deze niet killen?
\begin{solution}
ze krijgen \texttt{init} als parent
\end{solution}
\end{parts}

\question[10] Wie of wat heeft pid 0?
\begin{solution}
De kernel, ppid van init + kern threads = 0
\end{solution}


% tirlnx01 - Materiaal om het keuzevak Linux te geven 
% op de Hogeschool Rotterdam.
% Copyright (C) 2010 - 2011  Paul Sohier, Kevin van der Vlist
%
% This program is free software: you can redistribute it and/or modify
% it under the terms of the GNU General Public License as published by
% the Free Software Foundation, either version 3 of the License, or
% (at your option) any later version.
%
% This program is distributed in the hope that it will be useful,
% but WITHOUT ANY WARRANTY; without even the implied warranty of
% MERCHANTABILITY or FITNESS FOR A PARTICULAR PURPOSE.  See the
% GNU General Public License for more details.
%
% You should have received a copy of the GNU General Public License
% along with this program.  If not, see <http://www.gnu.org/licenses/>.
%
% Kevin van der Vlist - kevin@kevinvandervlist.nl
% Paul Sohier - paul@paulsohier.nl

\section{Week 4}
Deze vragen gaan over de volgende hoofdstukken:
\begin{itemize}
\item[1.] Linux networking
\item[2.] X Windows
\item[3.] Device files
\end{itemize}

\question[10] Geef het \emph{IP} en de interface naam van het netwerk device van de computer waar je nu op werkt. 
\begin{solution}
\begin{lstlisting}
kevin@slackbak:~$ /sbin/ifconfig 
eth0      Link encap:Ethernet  HWaddr 00:0c:29:fa:16:57  
          inet addr:145.24.222.162  Bcast:145.24.222.255  Mask:255.255.255.0
\end{lstlisting}%$
\end{solution}

\question[10] Geef het \emph{IP} adres en de \emph{RSA fingerprint} van de server van \emph{Slackware}, te vinden op ``slackware.org''. 
\begin{solution}
\begin{lstlisting}
kevin@slackbak:~$ ssh slackware.org
The authenticity of host 'slackware.org (168.150.251.105)' can't be established.
RSA key fingerprint is 87:92:7f:41:f2:4a:12:e7:93:97:70:85:cd:af:c8:b1.
Are you sure you want to continue connecting (yes/no)? ^C
\end{lstlisting}%$
\end{solution}

\question[10] Maak een bestand aan, en zet dit op de \emph{FTP} space van je studenten account. Zorg dat het in de ``public.www'' subdirectory staat. Tip: \emph{passive ftp} kan nodig zijn
\begin{solution}
\begin{lstlisting}
kevin@slackbak:~$ touch mijnupload
kevin@slackbak:~$ ftp -p ftp.hro.nl
Connected to orpheus.hro.nl.
[...]
220 Service Ready for new User
Name (ftp.hro.nl:kevin): 0814xxx.cmi
331 Password Needed for Login
Password:
230 User 0814xxx Logged in Successfully
Remote system type is NETWARE.
ftp> cd public.www
250 Directory successfully changed to "/USERS16/4/0814xxx/public.www"
ftp> put mijnupload bestand
local: mijnupload remote: bestand
200 PORT Command OK
150 Opening data connection for bestand (145.24.222.162,42544)
226 Transfer Complete
ftp> exit
221 Closing Session
\end{lstlisting}%$
\end{solution}

\question[10] Tijdens het programmeren van een stuk software is er een vereiste aan een hoge entropie om wachtwoorden te genereren. Onder geen beding mogen deze wachtwoorden voorspelbaar worden. Welk special device dient gebruikt te worden? Waarom?
\begin{solution}
random: blocking; hoge entropie\\
urandom: non-blocking; mogelijk voorspelbaar
\end{solution}

\question[10] Voeg de nameserver 2.4.8.16 toe. Geef de gedane stappen. 
\begin{solution}
\begin{lstlisting}
root@slackbak:/home/kevin# cat /etc/resolv.conf 
nameserver 2.4.8.16
\end{lstlisting}
\end{solution}

\question[10] Zorg dat wanneer er gezocht word naar de host \texttt{geheim.xyz}, dat dit resolved naar het \emph{IP} adres \texttt{1.2.3.4}. Geef de gedane stappen. 
\begin{solution}
\begin{lstlisting}
/etc/hosts
1.2.3.4          geheim.xyz
\end{lstlisting}
\end{solution}

\question[10] Maak een nieuw loopback device aan voor de \emph{IP} range \texttt{172.16.0.0}. Zorg dat de routing naar dit device verloo, en dat het adres \texttt{172.16.1.1} te pingen is. Verwijder de interface ook weer. Geef de commando's.
\begin{solution}
\begin{lstlisting}
ifconfig lo:0 172.16.0.0
route add 172.16.0.0 lo:0
ping 172.16.1.1
ifconfig lo:0 down
\end{lstlisting}
Note Paul: route lijkt niet nodig.
\end{solution}

\question[10] Verklaar de relatie tussen een X server en een X client. Leg uit wie de server, en wie de client is. Waarom is dit zo?
\begin{solution}

X server draait op de client, x client op de server (met networking) of client. Windows worden gepaint op de server, applicaties zijn zelf de client. Dit is gedaan vor de modulariteit en networking mogelijkheden
\end{solution}
\question[10] Veel window managers voor X zorgen dat applicaties als het ware in een frame worden geplaatst. Dit frame kan gebruikt worden voor de plaatsing van de window, maar bijvoorbeeld ook om verschillende buttons op te plaatsen. Hoe noemt men dit fenomeen?
\begin{solution}

Reparenting van windows
\end{solution}

\question[10] OPTIONEEL: Installeer een grafische omgeving, en configureer deze. Probeer dan eens van de desktop gebruik te maken. De voordelen van een krachtig, open systeem en een grafische omgeving voor standaard gebruik zullen dan duidelijk worden. Voor installatie van software, zie bijlage F en G van het dictaat.
\begin{solution}
Packages zijn het handigst
\end{solution}

% tirlnx01 - Materiaal om het keuzevak Linux te geven 
% op de Hogeschool Rotterdam.
% Copyright (C) 2010 - 2011  Paul Sohier, Kevin van der Vlist
%
% This program is free software: you can redistribute it and/or modify
% it under the terms of the GNU General Public License as published by
% the Free Software Foundation, either version 3 of the License, or
% (at your option) any later version.
%
% This program is distributed in the hope that it will be useful,
% but WITHOUT ANY WARRANTY; without even the implied warranty of
% MERCHANTABILITY or FITNESS FOR A PARTICULAR PURPOSE.  See the
% GNU General Public License for more details.
%
% You should have received a copy of the GNU General Public License
% along with this program.  If not, see <http://www.gnu.org/licenses/>.
%
% Kevin van der Vlist - kevin@kevinvandervlist.nl
% Paul Sohier - paul@paulsohier.nl

\section{Week 5}
Deze vragen gaan over de volgende hoofdstukken:
\begin{itemize}
\item[1.] Basis commando's
\item[2.] Shell
\end{itemize}

\question[5] Maak je eigen opdracht met behulp van opdrachtalliassering. De opdracht moet naar je home-directory springen en daar de pico editorstarten.
\begin{solution}
\begin{lstlisting}
kevin@slackbak:~$ tail -n 1 ~/.bash_profile
alias homepico='cd ~; pico'
\end{lstlisting}%$
\end{solution}

\question[0] Haal zoveel mogelijk informatie uit de volgende \texttt{bash} prompts:
\begin{parts}
\part[5] 
\begin{lstlisting}
kevin@slackbak:~$ 
\end{lstlisting}%$

\begin{solution}
user kevin @ host slackbak in homedir zonder speciale rechten
\end{solution}

\part[5]
\begin{lstlisting}
root@1.2.3.4:/home/# 
\end{lstlisting}

\begin{solution}

user root @ host 1.2.3.4 in /home/ met admin rechten
\end{solution}
\end{parts}

\question[10] Zoek naar alle bestanden met de extensie .txt en pak ze in met gzip. Maak gebruik van opdracht substitutie. %todo, check locatie in dictaat
\begin{solution}
\begin{lstlisting}
find / -name '*.txt' -exec cat {} \; | gzip > files.gz
\end{lstlisting}
\end{solution}

\question[0] Deze vraag gaat over cronjobs.
\begin{parts}
\part[10] Geef aan welke cron jobs er iedere dag gerund worden op een standaard Slackware systeem.
\begin{solution}
\begin{lstlisting}
root@slackbak /var/log# ls /etc/cron.daily/
certwatch  logrotate  slocate
\end{lstlisting}
\end{solution}

\part[10] Een van de standaard crons is logrotate. Hierdoor worden log files verplaatst en ingepakt, waardoor het systeem weer met lege files kan beginnen. Er worden een aantal van deze ingepakte logs bewaard. Is het verstandig om log files lang te bewaren op een server? Waarom wel of niet?
\begin{solution}
Nee, schijfruimte gebruik is hoog op een server met grote logs. 

Ja, schijfruimte weegt niet op tegen de hoeveelheid informatie die je hebt. 
\end{solution}
\end{parts}

\question[0] Vul de onderstaande kolom in voor de volgende situaties:
\begin{parts}
\part[10] Een \texttt{chmod 651}
\part[10] Een \texttt{(chmod 400; chmod +x)}
\part[10] Een \texttt{umask} van 022
\end{parts}
\begin{tabular}[t]{llll}
  Wat & a & b & c\\
  \hline
  user - read & \hspace{2 cm} & \hspace{2 cm} & \hspace{2 cm}\\
  user - write & \hspace{2 cm} & \hspace{2 cm} & \hspace{2 cm}\\
  user - execute & \hspace{2 cm} & \hspace{2 cm} & \hspace{2 cm}\\
  group - read & \hspace{2 cm} & \hspace{2 cm} & \hspace{2 cm}\\
  group - write & \hspace{2 cm} & \hspace{2 cm} & \hspace{2 cm}\\
  group - execute & \hspace{2 cm} & \hspace{2 cm} & \hspace{2 cm}\\
  other - read & \hspace{2 cm} & \hspace{2 cm} & \hspace{2 cm}\\
  other - write & \hspace{2 cm} & \hspace{2 cm} & \hspace{2 cm}\\
  other - execute & \hspace{2 cm} & \hspace{2 cm} & \hspace{2 cm}\\
\end{tabular}

\begin{solution}
\begin{tabular}[t]{llll}
  Wat & a & b & c\\
  \hline
  user - read & y & y & y\\
  user - write & y & - & y\\
  user - execute & - & y & y\\
  group - read & y & - & y\\
  group - write & - & - & -\\
  group - execute & y & y & y\\
  other - read & - & - & y\\
  other - write & - & - & -\\
  other - execute & y & y & y\\
\end{tabular}
\end{solution}

\question[10] Installeer de officiele slackware Java Development Kit package. Geef de uitvoer van \texttt{javac -version}. Geef de gedane stappen
\begin{solution}
\begin{lstlisting}
wget ftp://ftp.nluug.nl/pub/os/Linux/distr/slackware/slackware-13.1/extra/jdk-6/jdk-6u20-i586-1.txz
root@slackbak:/home/kevin# installpkg jdk-6u20-i586-1.txz 
Verifying package jdk-6u20-i586-1.txz.
Installing package jdk-6u20-i586-1.txz:
PACKAGE DESCRIPTION:
# Java(TM) 2 Platform Standard Edition Development Kit 6.0 update 20.
#
# The Java 2 SDK software includes tools for developing, testing, and
# running programs written in the Java programming language.  This
# package contains everything you need to run Java(TM).
#
# For additional information, refer to this Sun Microsystems web page:
#   http://java.sun.com/
#
Executing install script for jdk-6u20-i586-1.txz.
Package jdk-6u20-i586-1.txz installed.
root@slackbak:/home/kevin# /usr/lib/java/bin/javac -version
javac 1.6.0_20
\end{lstlisting}
\end{solution}

\question[0] Welk commando kan voor de volgende taken gebruikt worden:
\begin{parts}
\part[2\half] Bekijk de inhoud van een tekstbestand.
\begin{solution}
\texttt{cat}, \texttt{less}, \ldots
\end{solution}

\part[2\half] Een programma moet op vaste tijden worden uitgevoerd. Hoe is dit in te stellen?
\begin{solution}
\texttt{cron}
\end{solution}

\part[2\half] Zoek naar de tekenreeks \emph{DMA} in de file \emph{/var/log/dmesg}
\begin{solution}
\texttt{grep DMA /var/log/dmesg}
\end{solution}

\part[2\half] Een map genaamd \emph{current} moet altijd verwijzen naar de meest recente versie van software, zonder de map te kopieren. Hoe is dit te realiseren?
\begin{solution}
\texttt{ln -s prog-x.y current}
\end{solution}

\part[2\half] Een standard output stream bevat de tekenreeks \emph{abcde123}, wat een wachtwoord is. Deze tekenreeks dient gefilterd te worden naar \emph{geheim}. Tip: dit kan met de simpele expressie \texttt{'s/abcde123/geheim/'}.
\begin{solution}
\texttt{ | sed -e 's/abcde123/geheim/'}
\end{solution}

\part[2\half] Volg realtime een logfile op \texttt{/var/log/syslog}, zodat iedere nieuwe entry op de stdout geprint word. 
\begin{solution}
\texttt{tail -f /var/log/syslog}
\end{solution}
\end{parts}

% tirlnx01 - Materiaal om het keuzevak Linux te geven 
% op de Hogeschool Rotterdam.
% Copyright (C) 2010 - 2011  Paul Sohier, Kevin van der Vlist
%
% This program is free software: you can redistribute it and/or modify
% it under the terms of the GNU General Public License as published by
% the Free Software Foundation, either version 3 of the License, or
% (at your option) any later version.
%
% This program is distributed in the hope that it will be useful,
% but WITHOUT ANY WARRANTY; without even the implied warranty of
% MERCHANTABILITY or FITNESS FOR A PARTICULAR PURPOSE.  See the
% GNU General Public License for more details.
%
% You should have received a copy of the GNU General Public License
% along with this program.  If not, see <http://www.gnu.org/licenses/>.
%
% Kevin van der Vlist - kevin@kevinvandervlist.nl
% Paul Sohier - paul@paulsohier.nl

\section{Week 6}
Deze vragen gaan over de volgende hoofdstukken:
\begin{itemize}
\item[1.] Shell scripting
\end{itemize}

\question[15] Maak een script dat de gebruiker twee keer om invoer vraagt, waarna deze met elkaar worden vergeleken.
\begin{solution}
\begin{lstlisting}
#!/bin/bash
echo ``Geef twee waarde om te vergelijken''
read antwoord1
read antwoord2

if [ $antwoord == $antwoord ]; then
    echo ``Antwoord1 is gelijk aan antwoord1''
else
    echo ``Antwoord1 is niet gelijk aan antwoord2''
fi
\end{lstlisting}
\end{solution}

\question[15] Maak een case/switch die een string van de input leest, en switcht op groen, geel, blauw en een default.
\begin{solution}
\begin{lstlisting}
#!/bin/bash
echo ``Geef twee waarde om te vergelijken''
read input
case "$input" in 
  "groen")
    echo "groen" ;;
  "geel")
    echo "geel" ;;
  "blauw")
    echo "blauw" ;;
  *)
    echo "default" ;;
esac
\end{lstlisting}%$
\end{solution}

\question[0] Geef een oplossing voor de volgende tests. Zie de eerste vraag als voorbeeld
\begin{parts}
\part[5] Test of een string een lengte van 0 heeft.
\begin{lstlisting}
[ -z "$string" ]
\end{lstlisting}%$

\part[5] Test of de lengte van een string 1 of groter is. 
\begin{solution}
\begin{lstlisting}
[ -n "$string" ]
\end{lstlisting}%$
\end{solution}

\part[5] Kijk of integer a groter is dan integer b.
\begin{solution}
\begin{lstlisting}
[ "$a" -gt "$b" ]
\end{lstlisting}%$
\end{solution}

\part[5] Kijk of een bestand een symbolic link is.
\begin{solution}
\begin{lstlisting}
[ -h "$bestand" ]
\end{lstlisting}%$
\end{solution}

\part[5] Kijk of een bestand schrijfbaar is en groter is dan 0
\begin{solution}
\begin{lstlisting}
[ -w "$bestand" ] && [ -s "$bestand" ]
\end{lstlisting}%$
\end{solution}

\part[5] Kijk of een bestand bestaat en een bestand is. Ook moet er gekeken worden of a kleiner of gelijk is als b
\begin{solution}
\begin{lstlisting}
# stat -c %s geeft filesize
[ -f "$1" ] && [ $(stat -c%s $1) -le $(stat -c%s $2) ]
\end{lstlisting}%$
\end{solution}
\end{parts}

\question[20] Maak een script wat een directory als parameter neemt, en van alle files print hoeveel hardlinks het heeft. Voorbeeld output:
\begin{lstlisting}
/home/kevin/files.gz: 2
/home/kevin/fixemacs.sh: 1
\end{lstlisting}
\begin{solution}
\begin{lstlisting}
dir=/home/kevin; for i in $dir/*; do echo -n "$dir$i: "; stat -c %h $i; done
\end{lstlisting}%$
\end{solution}

\question[20] Installeer lighttpd (versie 1.4.28). Zorg ervoor dat er openssl, bzip2 en zlib support meegecompileerd word. Een configuratiebestand kan gevonden worden in de map \texttt{lighttpd-1.4.28/tests/lighttpd.conf}. Zorg dat de webserver kan starten, en dat deze luistert op port 8080. Waarom is de webserver niet te bereiken vanaf andere computers dan localhost? Geef de gedane stappen.

Tip: Kijk voor meer informatie in appendix G, Source Installatie.
\begin{solution}
\begin{lstlisting}
./configure --prefix=/home/kevin/software/lighthttpd --with-bzip2 --with-zlib --with-openssl
make
make install
export SRCDIR=/tmp
/home/kevin/software/lighthttpd/sbin/lighttpd -f /home/kevin/software/lighthttpd/etc/lighttpd.conf
## bind to port (default: 80)
server.port                 = 2048
## bind to localhost (default: all interfaces)
server.bind                = "localhost"
\end{lstlisting}
\end{solution}

% tirlnx01 - Materiaal om het keuzevak Linux te geven 
% op de Hogeschool Rotterdam.
% Copyright (C) 2010 - 2011  Paul Sohier, Kevin van der Vlist
%
% This program is free software: you can redistribute it and/or modify
% it under the terms of the GNU General Public License as published by
% the Free Software Foundation, either version 3 of the License, or
% (at your option) any later version.
%
% This program is distributed in the hope that it will be useful,
% but WITHOUT ANY WARRANTY; without even the implied warranty of
% MERCHANTABILITY or FITNESS FOR A PARTICULAR PURPOSE.  See the
% GNU General Public License for more details.
%
% You should have received a copy of the GNU General Public License
% along with this program.  If not, see <http://www.gnu.org/licenses/>.
%
% Kevin van der Vlist - kevin@kevinvandervlist.nl
% Paul Sohier - paul@paulsohier.nl

\section{Week 7}
Deze vragen gaan over de volgende hoofdstukken:
\begin{itemize}
\item[1.] Shell scripting
\end{itemize}

\question[10] Schrijf een script dat directories kan inlezen, en daarop volgordes kan sorteren (Bijvoorbeeld op grootte of alfabetische volgorde).
\begin{solution}
TODO
\end{solution}

\question[50] Maak met behulp van het case statement een menu dat weer bestaat uit verschillende opties, zoals:
    \begin{itemize}
      \item Een file in een variabele zetten.
      \item Een bestand kopieren
      \item Een bestand verplaatsen
      \item Een bestand aanpassen
      \item Een bestand emailen
      \item Een bestand verwijderen
    \end{itemize}
    Maak dit programma zo dat er om een wachtwoord gevraagd wordt bij het opstarten. Geeft de gebruiker een fout wachtwoord dan mag die persoon geen gebruik maken van het script. Zorg ervoor dat het wachtwoord wordt opgeslagen in een file (geheim.txt).
    Zorg ervoor dat alle menu opties ook werken (Eventueel met fake data, zoals bij email). Vraag om een bevestiging wanneer een actie niet ongedaan kan worden gemaakt. 

\textbf{Let op:} Doordat de mailserver niet is geconfigureerd zal je mailtje niet aankomen, het zal achter wel in de logfiles komen te staan. Je zal in de file \texttt{/var/log/maillog} dit zien:
\begin{lstlisting}
root@slackbak:/var/log# cat maillog
Mar 29 16:21:33 slackbak sendmail[2507]: p2TELXfV002507: from=paul, size=2524, class=0, nrcpts=1, msgid=<201103291421.p2TELXfV002507@slackbak.cmi-hro.nl>, relay=paul@localhost
Mar 29 16:21:33 slackbak sendmail[2507]: p2TELXfV002507: to=paul@hosthuis.nl, ctladdr=paul (1001/100), delay=00:00:00, xdelay=00:00:00, mailer=relay, pri=32524, relay=[127.0.0.1] [127.0.0.1], dsn=4.0.0, stat=Deferred: Connection refused by [127.0.0.1]
\end{lstlisting}

\begin{solution}
\begin{lstlisting}
#!/bin/bash

function menu
{
	echo "Kies een optie om uit te voeren: ";
	echo "1. Zet een bestand in een variable";
	echo "2. Kopieer een bestand naar een andere locatie";
	echo "3. Verplaats een bestand naar een andere locatie";
	echo "4. Pas een bestand aan via een editor";
	echo "5. Email een bestand naar een gebruiker";
	echo "6. Verwijder een bestand";
	
	echo "";
	
	echo "Voer het nummer in van wat je wilt, of q om af te sluiten";
}

function zetInVariable
{
	echo "Welke file wil je in een variable zetten?";
	read file;

	if [ -e $file ] ; then
		echo "We gaan nu ${file} in een variable zetten.";

#		$newData=`cat `;

		# Debug:
		echo $newData;
	else
		echo "${file} bestaat niet? ";
	fi

	sleep 3; # Even slapen zodat gebruiker rustig kan lezen
}

function kopieer
{
    echo "Welke file wil je kopieren?";
    read file
    echo "waarheen wil je hem kopieren?";
    read loc

    if [ -e $file ] ; then
	cp $file $loc
    else
	echo "${file} bestaat niet";
    fi
    sleep 3; # Even slapen;
}

function verplaats
{
    echo "welke file wil je verplaatsen?";

    read file
    echo "waarheen wil je hem verplaatsen?";
    read loc

    if [ -e $file ] ; then
	mv $file $loc
    else
	echo "${file} bestaat niet.";
    fi

    sleep 3;
}

function verwijder
{
    echo "Welke file wil je verwijderen?";

    read file

    echo "Weet je zeker dat je ${file} wilt verwijderen? Type J.";
    read ja

    if [ $ja == "J" ] ; then
	rm $file;
	echo "Verwijderd";
    fi
    sleep 3;
}

function edit
{
    echo "Welke file wil je aanpassen?";
    read file

    `emacs ${file}`;
}

function email
{
    echo "Welke file wil je sturen?";
    read file;

    echo "Waarheen wil je hem sturen?";
    read email;

    echo  "Welk onderwerp wil je hem geven?";
    read onder;

    if [ -e $file ]; then
	cat $file | /usr/bin/mail -s "$onder" "$email";
	echo "Verzonden";
    else
	echo "${file} bestaat niet.";
    fi
}

exit=0;
until [ $exit -gt 1 ]; do
      menu
      read optie
      echo "Gekozen voor ${optie}";
	case $optie in
	     "q") exit;;
	     "1") zetInVariable;;
	     "2") kopieer;;
	     "3") verplaats;;
	     "4") edit;;
	     "5") email;;
	     "6") verwijder;;

	esac
 done
\end{lstlisting}%$
\end{solution}

\question[20] Maak een script wat een gebruikersnaam of uid accepteert als parameter. Daarna zal het alle groepen printen waar deze gebruiker lid van is. 
\begin{solution}
\begin{lstlisting}
#!/bin/bash
function getNameFromUID {
    local LINE=`grep -E ".+:.+:$1:[0-9]+.+" /etc/passwd`
    for x in $LINE; do
        NAME=$x
        break
    done
}

function getGroupsFromName {
    for GROUP in `grep $1 /etc/group | perl -pe 's/([a-z]+):(.+)/\1/'`; do
        echo $GROUP
    done
}

if [ "$#" -eq "0" ]; then
    echo "Geef een gebruikersnaam of UID op."
    echo "./script.sh kevin"
    exit 1
fi

# Internal field seperator - bash shell built-in
OIFS=$IFS
IFS=':'

if [[ "$1" =~ ^[0-9]+$ ]]; then
    getNameFromUID $1
else
    NAME=$1
fi

echo "User $NAME zit in de volgende groepen:"
getGroupsFromName $NAME
IFS=$OIFS
exit 0
\end{lstlisting}
\end{solution}

\question[20] Maak een script wat backups maakt. Zoek zelf uit hoe. Enige eis is dat het script de directorie(s) (Er kunnen dus meerdere parameters opgegeven worden!) welke gebackuped moeten worden opgegeven kan worden als parameter. Bij het inleveren dien je uit te leggen waarom je welke keuzes hebt gemaakt.
\begin{solution}
\begin{lstlisting}
tar -cjf /backup/file.tar.bz2 /home/kevin/ /home/paul/
\end{lstlisting}
\end{solution}


% Eindigen van questions
\end{questions}
% Einde document
\end{document}
