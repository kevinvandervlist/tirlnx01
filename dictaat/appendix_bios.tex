% tirlnx01 - Materiaal om het keuzevak Linux te geven 
% op de Hogeschool Rotterdam.
% Copyright (C) 2010 - 2011  Paul Sohier, Kevin van der Vlist
%
% This program is free software: you can redistribute it and/or modify
% it under the terms of the GNU General Public License as published by
% the Free Software Foundation, either version 3 of the License, or
% (at your option) any later version.
%
% This program is distributed in the hope that it will be useful,
% but WITHOUT ANY WARRANTY; without even the implied warranty of
% MERCHANTABILITY or FITNESS FOR A PARTICULAR PURPOSE.  See the
% GNU General Public License for more details.
%
% You should have received a copy of the GNU General Public License
% along with this program.  If not, see <http://www.gnu.org/licenses/>.
%
% Kevin van der Vlist - kevin@kevinvandervlist.nl
% Paul Sohier - paul@paulsohier.nl

\chapter{BIOS}
De BIOS is een belangrijk onderdeel van je computer. BIOS staat voor \emph{Basic Input Output System}. Zonder BIOS zal je PC nooit werken. Wanneer je je PC opstart is het eerste wat je ziet je BIOS, nog voordat het operating system gestart is. De BIOS zal je basis hardware controleren en regelt alle communcatie tussen het operating system en de hardware. 

In elke BIOS zijn instellingen aan te passen, en vaak heb je deze nodig om bijvoorbeeld Linux te kunnen installeren. 

\section{Boot sequence}
Om Linux te kunnen installeren moet er vaak vanaf een ander medium als de aanwezige harde schijf opgestart worden. In veel gevallen is dit echter standaard uitgeschakeld en moet je de BIOS instellingen aanpassen naar je eigen wens zodat je van bijvoorbeeld CD of USB stick kan booten.
Om dit te doen moet je opstarten in de BIOS. Vaak doe je dit door een toets zoals F2 of del in te drukken, welke dit is hangt af van de gebruikte hardware en wordt vaak weergeven bij het opstarten.

Zodra je in de BIOS bent zal je op zoek moeten gaan naar de plek waar je de boot sequence aan kan passen. Vaak is dit een apart tabblad dat Boot Sequence heet. In dit tabblad zal je de sequence zo moeten aanpassen dat jou installatie medium voor de harde schijf staat. Zodra je dit gedaan hebt kan je het opslaan en opnieuw opstarten. Als het goed is start je PC nu op vanaf jou installatie medium.
